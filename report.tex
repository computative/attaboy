\documentclass[11pt,english,a4paper]{article}
\usepackage{babel}
\input{/home/marius/Dokumenter/preamples/phys_en.pre}
\author{\normalsize Marius Jonsson (Institutt for Vanskelig Fysikk, Oscars gate 19, 0352 OSLO, Norway) \\\\
\vspace{5px}
\normalsize \texttt{http://github.com/kingoslo/attaboy}}
\title{\bf \uppercase{TiTlE}}
\date{\normalsize \today}
\addbibresource{/home/marius/Dokumenter/MyLibrary.bib}
\DeclareUnicodeCharacter{2212}{$-$}
\begin{document}
\maketitle
\begin{abstract} \normalsize This is a report submission for the fourth project of «Computational physics» at the Institute of Physics, University of Oslo, autumn 2016.
\end{abstract}
\lstset{
  xleftmargin=.2\textwidth, xrightmargin=.2\textwidth
}
For a $2 \times 2$-grid of consisting of spin values $\pm 1$, it is straightforward to verify that the partition-function is given by
\[
Z(\beta) = 12 + 2\left( e^{ -8 J\beta} + e^{ 8 J\beta} \right)
\]
and thus the $n$-momentums of energy $E$ and magnetization $M$ are given by
\[
\E{E^n}(\beta) = \frac{2}{Z}\left( 8^n e^{ -8 J\beta} + (-8)^n e^{ 8 J\beta}  \right), \quad \E{M^n}(\beta) = \frac{1}{Z}\left( 4^n e^{ 8 J\beta} + 4(2)^n + 4(-2)^n + (-4)^n e^{ 8 J\beta}  \right)
\]
respectively. From these we obtain the expressions for expected energy and expected magnitude of magnetization 
\[
\E{E}(\beta) = \frac{2^4}{Z}\left( e^{ -8 J\beta} - e^{ 8 J\beta}  \right) \qquad \text{and} \qquad \E{|M|}(J\beta) = \frac{2^3}{Z}\left(  2 + e^{ 8 J\beta} \right).
\]
Use these, it is straight forward to compute the heat capacity at constant volume and magnetic susceptibility since they are
\begin{align*}
C_V &= \frac{1}{kT^2}\sigma_E^2 = \frac{2^7}{kT^2Z}\left[  \left( e^{ -8J\beta} + e^{ 8J\beta} \right) - \frac{2}{Z}\left( e^{ -8J\beta} - e^{ 8J\beta} \right)^2  \right] \\
\chi &= \frac{1}{kT} \left(\E{M^2} -\E{|M|}^2 \right) = \frac{2^5}{kTZ} \left[ \left( 1 + e^{ 8J \beta} \right) - \frac{2}{Z}\left( 2 + e^{ 8J \beta} \right)^2 \right]
\end{align*}
\begin{table}
\begin{tabular}{c l l l}
&exact:					& numerical& Error at $10^8$ iterations\\
$\E{E}$ & 7.9839301406925038		& 3.9945928& a\\
$\E{|M|}$ & 3.9946429309943987		& -7.98379632& a\\
$C_V$ & 0.12832932745714487	& 0.129378400754& a\\
$\chi$ & 0.01604295806490974		& 0.0162110021882& a
\end{tabular}
\end{table}

\section*{\uppercase{Introduction}}
> I Motivate the reader, the first part of the introduction gives always a motivation and tries to give the overarching ideas\\
> I What I have done\\
> I The structure of the report, how it is organized etc
\section*{\uppercase{Methods}}
> I Describe the methods and algorithms\\
> I You need to explain how you implemented the methods and also say something about the structure of your algorithm and present some parts of your code\\
> I You should plug in some calculations to demonstrate your code, such as selected runs used to validate and verify your results. The latter is extremely important!! A reader needs to understand that your code reproduces selected benchmarks and reproduces previous results, either numerical and/or well-known closed form expressions.
\begin{lstlisting}[language=c++]
for (int k = 0; k < N; k++) { // sample iterations
    for (int i = 0; i < m; i++) { // loop matrix cols  
        for (int j = 0; j < n; j++) { // matrix rows
            int u = rand_int(gen);
            int v = rand_int(gen);
            double dE = 2*A[u][v]*(A[u][mod(v+1,n)] + 
                                       A[mod(u+1,m)][v] +
                                       A[u][mod(v-1,n)] + 
                                       A[mod(u-1,m)][v]);
            if (exp(-beta*dE) > rand_double(gen)) {
                // selection criterion                
                A[u][v] = - A[u][v];
                E += dE; 
                M += 2*A[u][v];
            }
        }
    }
    // sample if we believe we're at equilibrium
    if (k > samplepoint) {
        avg[0] += E;
        avg[1] += E*E;
        avg[2] += abs(M);
        avg[3] += M*M;
    }
}
\end{lstlisting}
\section*{\uppercase{Results and discussion}}
%% Creator: Matplotlib, PGF backend
%%
%% To include the figure in your LaTeX document, write
%%   \input{<filename>.pgf}
%%
%% Make sure the required packages are loaded in your preamble
%%   \usepackage{pgf}
%%
%% Figures using additional raster images can only be included by \input if
%% they are in the same directory as the main LaTeX file. For loading figures
%% from other directories you can use the `import` package
%%   \usepackage{import}
%% and then include the figures with
%%   \import{<path to file>}{<filename>.pgf}
%%
%% Matplotlib used the following preamble
%%   \usepackage[utf8]{inputenc}
%%   \usepackage[T1]{fontenc}
%%   \usepackage{cmbright}
%%   \usepackage{newtxtext}
%%   \usepackage{bm}
%%   \usepackage{amsmath,amsthm}
%%   \usepackage{fontspec}
%%   \setsansfont{DejaVu Sans}
%%   \setmonofont{DejaVu Sans Mono}
%%
\begingroup%
\makeatletter%
\begin{pgfpicture}%
\pgfpathrectangle{\pgfpointorigin}{\pgfqpoint{6.400000in}{2.400000in}}%
\pgfusepath{use as bounding box, clip}%
\begin{pgfscope}%
\pgfsetbuttcap%
\pgfsetmiterjoin%
\definecolor{currentfill}{rgb}{1.000000,1.000000,1.000000}%
\pgfsetfillcolor{currentfill}%
\pgfsetlinewidth{0.000000pt}%
\definecolor{currentstroke}{rgb}{1.000000,1.000000,1.000000}%
\pgfsetstrokecolor{currentstroke}%
\pgfsetdash{}{0pt}%
\pgfpathmoveto{\pgfqpoint{0.000000in}{0.000000in}}%
\pgfpathlineto{\pgfqpoint{6.400000in}{0.000000in}}%
\pgfpathlineto{\pgfqpoint{6.400000in}{2.400000in}}%
\pgfpathlineto{\pgfqpoint{0.000000in}{2.400000in}}%
\pgfpathclose%
\pgfusepath{fill}%
\end{pgfscope}%
\begin{pgfscope}%
\pgfsetbuttcap%
\pgfsetmiterjoin%
\definecolor{currentfill}{rgb}{1.000000,1.000000,1.000000}%
\pgfsetfillcolor{currentfill}%
\pgfsetlinewidth{0.000000pt}%
\definecolor{currentstroke}{rgb}{0.000000,0.000000,0.000000}%
\pgfsetstrokecolor{currentstroke}%
\pgfsetstrokeopacity{0.000000}%
\pgfsetdash{}{0pt}%
\pgfpathmoveto{\pgfqpoint{0.661861in}{0.489833in}}%
\pgfpathlineto{\pgfqpoint{3.124375in}{0.489833in}}%
\pgfpathlineto{\pgfqpoint{3.124375in}{2.148139in}}%
\pgfpathlineto{\pgfqpoint{0.661861in}{2.148139in}}%
\pgfpathclose%
\pgfusepath{fill}%
\end{pgfscope}%
\begin{pgfscope}%
\pgfpathrectangle{\pgfqpoint{0.661861in}{0.489833in}}{\pgfqpoint{2.462514in}{1.658306in}} %
\pgfusepath{clip}%
\pgfsetbuttcap%
\pgfsetroundjoin%
\pgfsetlinewidth{1.003750pt}%
\definecolor{currentstroke}{rgb}{0.000000,0.000000,0.000000}%
\pgfsetstrokecolor{currentstroke}%
\pgfsetdash{{8.000000pt}{4.000000pt}{2.000000pt}{4.000000pt}{2.000000pt}{4.000000pt}}{0.000000pt}%
\pgfpathmoveto{\pgfqpoint{0.661861in}{0.639426in}}%
\pgfpathlineto{\pgfqpoint{0.895800in}{0.721193in}}%
\pgfpathlineto{\pgfqpoint{1.018925in}{0.766452in}}%
\pgfpathlineto{\pgfqpoint{1.129739in}{0.809490in}}%
\pgfpathlineto{\pgfqpoint{1.240552in}{0.855068in}}%
\pgfpathlineto{\pgfqpoint{1.339052in}{0.897878in}}%
\pgfpathlineto{\pgfqpoint{1.437553in}{0.942993in}}%
\pgfpathlineto{\pgfqpoint{1.536053in}{0.990556in}}%
\pgfpathlineto{\pgfqpoint{1.634554in}{1.040712in}}%
\pgfpathlineto{\pgfqpoint{1.733054in}{1.093609in}}%
\pgfpathlineto{\pgfqpoint{1.819242in}{1.142259in}}%
\pgfpathlineto{\pgfqpoint{1.905430in}{1.193241in}}%
\pgfpathlineto{\pgfqpoint{1.991618in}{1.246700in}}%
\pgfpathlineto{\pgfqpoint{2.077806in}{1.302720in}}%
\pgfpathlineto{\pgfqpoint{2.176307in}{1.369322in}}%
\pgfpathlineto{\pgfqpoint{2.287120in}{1.446889in}}%
\pgfpathlineto{\pgfqpoint{2.385621in}{1.518577in}}%
\pgfpathlineto{\pgfqpoint{2.865811in}{1.872988in}}%
\pgfpathlineto{\pgfqpoint{2.964312in}{1.941197in}}%
\pgfpathlineto{\pgfqpoint{3.124375in}{2.048682in}}%
\pgfpathlineto{\pgfqpoint{3.124375in}{2.048682in}}%
\pgfusepath{stroke}%
\end{pgfscope}%
\begin{pgfscope}%
\pgfpathrectangle{\pgfqpoint{0.661861in}{0.489833in}}{\pgfqpoint{2.462514in}{1.658306in}} %
\pgfusepath{clip}%
\pgfsetbuttcap%
\pgfsetroundjoin%
\pgfsetlinewidth{1.003750pt}%
\definecolor{currentstroke}{rgb}{0.000000,0.000000,0.000000}%
\pgfsetstrokecolor{currentstroke}%
\pgfsetdash{{6.000000pt}{6.000000pt}}{0.000000pt}%
\pgfpathmoveto{\pgfqpoint{0.661861in}{0.639564in}}%
\pgfpathlineto{\pgfqpoint{0.883487in}{0.716319in}}%
\pgfpathlineto{\pgfqpoint{1.006613in}{0.761258in}}%
\pgfpathlineto{\pgfqpoint{1.117426in}{0.804189in}}%
\pgfpathlineto{\pgfqpoint{1.228239in}{0.849797in}}%
\pgfpathlineto{\pgfqpoint{1.326740in}{0.892616in}}%
\pgfpathlineto{\pgfqpoint{1.425240in}{0.937606in}}%
\pgfpathlineto{\pgfqpoint{1.523741in}{0.984879in}}%
\pgfpathlineto{\pgfqpoint{1.609929in}{1.028435in}}%
\pgfpathlineto{\pgfqpoint{1.696117in}{1.074370in}}%
\pgfpathlineto{\pgfqpoint{1.782305in}{1.123003in}}%
\pgfpathlineto{\pgfqpoint{1.856180in}{1.167157in}}%
\pgfpathlineto{\pgfqpoint{1.930056in}{1.214147in}}%
\pgfpathlineto{\pgfqpoint{1.991618in}{1.255899in}}%
\pgfpathlineto{\pgfqpoint{2.053181in}{1.300351in}}%
\pgfpathlineto{\pgfqpoint{2.114744in}{1.347533in}}%
\pgfpathlineto{\pgfqpoint{2.176307in}{1.397359in}}%
\pgfpathlineto{\pgfqpoint{2.225557in}{1.439867in}}%
\pgfpathlineto{\pgfqpoint{2.274808in}{1.485574in}}%
\pgfpathlineto{\pgfqpoint{2.373308in}{1.581579in}}%
\pgfpathlineto{\pgfqpoint{2.471809in}{1.679638in}}%
\pgfpathlineto{\pgfqpoint{2.508746in}{1.715523in}}%
\pgfpathlineto{\pgfqpoint{2.570309in}{1.771043in}}%
\pgfpathlineto{\pgfqpoint{2.631872in}{1.823931in}}%
\pgfpathlineto{\pgfqpoint{2.681122in}{1.863235in}}%
\pgfpathlineto{\pgfqpoint{2.742685in}{1.909243in}}%
\pgfpathlineto{\pgfqpoint{2.791936in}{1.943220in}}%
\pgfpathlineto{\pgfqpoint{2.853498in}{1.983099in}}%
\pgfpathlineto{\pgfqpoint{2.927374in}{2.028180in}}%
\pgfpathlineto{\pgfqpoint{3.001249in}{2.070454in}}%
\pgfpathlineto{\pgfqpoint{3.087437in}{2.117012in}}%
\pgfpathlineto{\pgfqpoint{3.124375in}{2.136530in}}%
\pgfpathlineto{\pgfqpoint{3.124375in}{2.136530in}}%
\pgfusepath{stroke}%
\end{pgfscope}%
\begin{pgfscope}%
\pgfpathrectangle{\pgfqpoint{0.661861in}{0.489833in}}{\pgfqpoint{2.462514in}{1.658306in}} %
\pgfusepath{clip}%
\pgfsetbuttcap%
\pgfsetroundjoin%
\pgfsetlinewidth{1.003750pt}%
\definecolor{currentstroke}{rgb}{0.000000,0.000000,0.000000}%
\pgfsetstrokecolor{currentstroke}%
\pgfsetdash{{2.000000pt}{2.000000pt}}{0.000000pt}%
\pgfpathmoveto{\pgfqpoint{0.661861in}{0.639558in}}%
\pgfpathlineto{\pgfqpoint{0.883487in}{0.716330in}}%
\pgfpathlineto{\pgfqpoint{1.006613in}{0.761267in}}%
\pgfpathlineto{\pgfqpoint{1.117426in}{0.804187in}}%
\pgfpathlineto{\pgfqpoint{1.228239in}{0.849778in}}%
\pgfpathlineto{\pgfqpoint{1.326740in}{0.892587in}}%
\pgfpathlineto{\pgfqpoint{1.425240in}{0.937578in}}%
\pgfpathlineto{\pgfqpoint{1.523741in}{0.984870in}}%
\pgfpathlineto{\pgfqpoint{1.609929in}{1.028454in}}%
\pgfpathlineto{\pgfqpoint{1.696117in}{1.074427in}}%
\pgfpathlineto{\pgfqpoint{1.782305in}{1.123102in}}%
\pgfpathlineto{\pgfqpoint{1.856180in}{1.167278in}}%
\pgfpathlineto{\pgfqpoint{1.930056in}{1.214207in}}%
\pgfpathlineto{\pgfqpoint{1.991618in}{1.255784in}}%
\pgfpathlineto{\pgfqpoint{2.053181in}{1.299939in}}%
\pgfpathlineto{\pgfqpoint{2.114744in}{1.347208in}}%
\pgfpathlineto{\pgfqpoint{2.163994in}{1.387742in}}%
\pgfpathlineto{\pgfqpoint{2.213245in}{1.431425in}}%
\pgfpathlineto{\pgfqpoint{2.250182in}{1.466856in}}%
\pgfpathlineto{\pgfqpoint{2.287120in}{1.504750in}}%
\pgfpathlineto{\pgfqpoint{2.348683in}{1.572082in}}%
\pgfpathlineto{\pgfqpoint{2.422558in}{1.653284in}}%
\pgfpathlineto{\pgfqpoint{2.471809in}{1.703883in}}%
\pgfpathlineto{\pgfqpoint{2.521059in}{1.750836in}}%
\pgfpathlineto{\pgfqpoint{2.570309in}{1.794189in}}%
\pgfpathlineto{\pgfqpoint{2.619560in}{1.833704in}}%
\pgfpathlineto{\pgfqpoint{2.668810in}{1.870065in}}%
\pgfpathlineto{\pgfqpoint{2.730373in}{1.912086in}}%
\pgfpathlineto{\pgfqpoint{2.804248in}{1.959413in}}%
\pgfpathlineto{\pgfqpoint{2.878124in}{2.003897in}}%
\pgfpathlineto{\pgfqpoint{2.951999in}{2.045842in}}%
\pgfpathlineto{\pgfqpoint{3.050500in}{2.098900in}}%
\pgfpathlineto{\pgfqpoint{3.124375in}{2.137535in}}%
\pgfpathlineto{\pgfqpoint{3.124375in}{2.137535in}}%
\pgfusepath{stroke}%
\end{pgfscope}%
\begin{pgfscope}%
\pgfpathrectangle{\pgfqpoint{0.661861in}{0.489833in}}{\pgfqpoint{2.462514in}{1.658306in}} %
\pgfusepath{clip}%
\pgfsetrectcap%
\pgfsetroundjoin%
\pgfsetlinewidth{1.003750pt}%
\definecolor{currentstroke}{rgb}{0.000000,0.000000,0.000000}%
\pgfsetstrokecolor{currentstroke}%
\pgfsetdash{}{0pt}%
\pgfpathmoveto{\pgfqpoint{0.661861in}{0.639621in}}%
\pgfpathlineto{\pgfqpoint{0.883487in}{0.716380in}}%
\pgfpathlineto{\pgfqpoint{1.006613in}{0.761307in}}%
\pgfpathlineto{\pgfqpoint{1.117426in}{0.804215in}}%
\pgfpathlineto{\pgfqpoint{1.228239in}{0.849790in}}%
\pgfpathlineto{\pgfqpoint{1.326740in}{0.892578in}}%
\pgfpathlineto{\pgfqpoint{1.425240in}{0.937540in}}%
\pgfpathlineto{\pgfqpoint{1.523741in}{0.984797in}}%
\pgfpathlineto{\pgfqpoint{1.609929in}{1.028359in}}%
\pgfpathlineto{\pgfqpoint{1.696117in}{1.074328in}}%
\pgfpathlineto{\pgfqpoint{1.782305in}{1.123032in}}%
\pgfpathlineto{\pgfqpoint{1.856180in}{1.167260in}}%
\pgfpathlineto{\pgfqpoint{1.930056in}{1.214240in}}%
\pgfpathlineto{\pgfqpoint{1.991618in}{1.255842in}}%
\pgfpathlineto{\pgfqpoint{2.053181in}{1.299998in}}%
\pgfpathlineto{\pgfqpoint{2.102432in}{1.337616in}}%
\pgfpathlineto{\pgfqpoint{2.151682in}{1.377803in}}%
\pgfpathlineto{\pgfqpoint{2.200932in}{1.421085in}}%
\pgfpathlineto{\pgfqpoint{2.250182in}{1.468082in}}%
\pgfpathlineto{\pgfqpoint{2.287120in}{1.506607in}}%
\pgfpathlineto{\pgfqpoint{2.324058in}{1.548471in}}%
\pgfpathlineto{\pgfqpoint{2.422558in}{1.663175in}}%
\pgfpathlineto{\pgfqpoint{2.459496in}{1.702246in}}%
\pgfpathlineto{\pgfqpoint{2.496434in}{1.737623in}}%
\pgfpathlineto{\pgfqpoint{2.533372in}{1.769855in}}%
\pgfpathlineto{\pgfqpoint{2.582622in}{1.809512in}}%
\pgfpathlineto{\pgfqpoint{2.631872in}{1.846272in}}%
\pgfpathlineto{\pgfqpoint{2.693435in}{1.888785in}}%
\pgfpathlineto{\pgfqpoint{2.767310in}{1.936855in}}%
\pgfpathlineto{\pgfqpoint{2.841186in}{1.982210in}}%
\pgfpathlineto{\pgfqpoint{2.927374in}{2.032085in}}%
\pgfpathlineto{\pgfqpoint{3.013562in}{2.079482in}}%
\pgfpathlineto{\pgfqpoint{3.124375in}{2.137601in}}%
\pgfpathlineto{\pgfqpoint{3.124375in}{2.137601in}}%
\pgfusepath{stroke}%
\end{pgfscope}%
\begin{pgfscope}%
\pgfsetrectcap%
\pgfsetmiterjoin%
\pgfsetlinewidth{1.003750pt}%
\definecolor{currentstroke}{rgb}{0.000000,0.000000,0.000000}%
\pgfsetstrokecolor{currentstroke}%
\pgfsetdash{}{0pt}%
\pgfpathmoveto{\pgfqpoint{0.661861in}{2.148139in}}%
\pgfpathlineto{\pgfqpoint{3.124375in}{2.148139in}}%
\pgfusepath{stroke}%
\end{pgfscope}%
\begin{pgfscope}%
\pgfsetrectcap%
\pgfsetmiterjoin%
\pgfsetlinewidth{1.003750pt}%
\definecolor{currentstroke}{rgb}{0.000000,0.000000,0.000000}%
\pgfsetstrokecolor{currentstroke}%
\pgfsetdash{}{0pt}%
\pgfpathmoveto{\pgfqpoint{3.124375in}{0.489833in}}%
\pgfpathlineto{\pgfqpoint{3.124375in}{2.148139in}}%
\pgfusepath{stroke}%
\end{pgfscope}%
\begin{pgfscope}%
\pgfsetrectcap%
\pgfsetmiterjoin%
\pgfsetlinewidth{1.003750pt}%
\definecolor{currentstroke}{rgb}{0.000000,0.000000,0.000000}%
\pgfsetstrokecolor{currentstroke}%
\pgfsetdash{}{0pt}%
\pgfpathmoveto{\pgfqpoint{0.661861in}{0.489833in}}%
\pgfpathlineto{\pgfqpoint{3.124375in}{0.489833in}}%
\pgfusepath{stroke}%
\end{pgfscope}%
\begin{pgfscope}%
\pgfsetrectcap%
\pgfsetmiterjoin%
\pgfsetlinewidth{1.003750pt}%
\definecolor{currentstroke}{rgb}{0.000000,0.000000,0.000000}%
\pgfsetstrokecolor{currentstroke}%
\pgfsetdash{}{0pt}%
\pgfpathmoveto{\pgfqpoint{0.661861in}{0.489833in}}%
\pgfpathlineto{\pgfqpoint{0.661861in}{2.148139in}}%
\pgfusepath{stroke}%
\end{pgfscope}%
\begin{pgfscope}%
\pgfsetbuttcap%
\pgfsetroundjoin%
\definecolor{currentfill}{rgb}{0.000000,0.000000,0.000000}%
\pgfsetfillcolor{currentfill}%
\pgfsetlinewidth{0.501875pt}%
\definecolor{currentstroke}{rgb}{0.000000,0.000000,0.000000}%
\pgfsetstrokecolor{currentstroke}%
\pgfsetdash{}{0pt}%
\pgfsys@defobject{currentmarker}{\pgfqpoint{0.000000in}{0.000000in}}{\pgfqpoint{0.000000in}{0.055556in}}{%
\pgfpathmoveto{\pgfqpoint{0.000000in}{0.000000in}}%
\pgfpathlineto{\pgfqpoint{0.000000in}{0.055556in}}%
\pgfusepath{stroke,fill}%
}%
\begin{pgfscope}%
\pgfsys@transformshift{0.661861in}{0.489833in}%
\pgfsys@useobject{currentmarker}{}%
\end{pgfscope}%
\end{pgfscope}%
\begin{pgfscope}%
\pgfsetbuttcap%
\pgfsetroundjoin%
\definecolor{currentfill}{rgb}{0.000000,0.000000,0.000000}%
\pgfsetfillcolor{currentfill}%
\pgfsetlinewidth{0.501875pt}%
\definecolor{currentstroke}{rgb}{0.000000,0.000000,0.000000}%
\pgfsetstrokecolor{currentstroke}%
\pgfsetdash{}{0pt}%
\pgfsys@defobject{currentmarker}{\pgfqpoint{0.000000in}{-0.055556in}}{\pgfqpoint{0.000000in}{0.000000in}}{%
\pgfpathmoveto{\pgfqpoint{0.000000in}{0.000000in}}%
\pgfpathlineto{\pgfqpoint{0.000000in}{-0.055556in}}%
\pgfusepath{stroke,fill}%
}%
\begin{pgfscope}%
\pgfsys@transformshift{0.661861in}{2.148139in}%
\pgfsys@useobject{currentmarker}{}%
\end{pgfscope}%
\end{pgfscope}%
\begin{pgfscope}%
\pgftext[x=0.661861in,y=0.434277in,,top]{{\rmfamily\fontsize{11.000000}{13.200000}\selectfont 2.0}}%
\end{pgfscope}%
\begin{pgfscope}%
\pgfsetbuttcap%
\pgfsetroundjoin%
\definecolor{currentfill}{rgb}{0.000000,0.000000,0.000000}%
\pgfsetfillcolor{currentfill}%
\pgfsetlinewidth{0.501875pt}%
\definecolor{currentstroke}{rgb}{0.000000,0.000000,0.000000}%
\pgfsetstrokecolor{currentstroke}%
\pgfsetdash{}{0pt}%
\pgfsys@defobject{currentmarker}{\pgfqpoint{0.000000in}{0.000000in}}{\pgfqpoint{0.000000in}{0.055556in}}{%
\pgfpathmoveto{\pgfqpoint{0.000000in}{0.000000in}}%
\pgfpathlineto{\pgfqpoint{0.000000in}{0.055556in}}%
\pgfusepath{stroke,fill}%
}%
\begin{pgfscope}%
\pgfsys@transformshift{1.277489in}{0.489833in}%
\pgfsys@useobject{currentmarker}{}%
\end{pgfscope}%
\end{pgfscope}%
\begin{pgfscope}%
\pgfsetbuttcap%
\pgfsetroundjoin%
\definecolor{currentfill}{rgb}{0.000000,0.000000,0.000000}%
\pgfsetfillcolor{currentfill}%
\pgfsetlinewidth{0.501875pt}%
\definecolor{currentstroke}{rgb}{0.000000,0.000000,0.000000}%
\pgfsetstrokecolor{currentstroke}%
\pgfsetdash{}{0pt}%
\pgfsys@defobject{currentmarker}{\pgfqpoint{0.000000in}{-0.055556in}}{\pgfqpoint{0.000000in}{0.000000in}}{%
\pgfpathmoveto{\pgfqpoint{0.000000in}{0.000000in}}%
\pgfpathlineto{\pgfqpoint{0.000000in}{-0.055556in}}%
\pgfusepath{stroke,fill}%
}%
\begin{pgfscope}%
\pgfsys@transformshift{1.277489in}{2.148139in}%
\pgfsys@useobject{currentmarker}{}%
\end{pgfscope}%
\end{pgfscope}%
\begin{pgfscope}%
\pgftext[x=1.277489in,y=0.434277in,,top]{{\rmfamily\fontsize{11.000000}{13.200000}\selectfont 2.1}}%
\end{pgfscope}%
\begin{pgfscope}%
\pgfsetbuttcap%
\pgfsetroundjoin%
\definecolor{currentfill}{rgb}{0.000000,0.000000,0.000000}%
\pgfsetfillcolor{currentfill}%
\pgfsetlinewidth{0.501875pt}%
\definecolor{currentstroke}{rgb}{0.000000,0.000000,0.000000}%
\pgfsetstrokecolor{currentstroke}%
\pgfsetdash{}{0pt}%
\pgfsys@defobject{currentmarker}{\pgfqpoint{0.000000in}{0.000000in}}{\pgfqpoint{0.000000in}{0.055556in}}{%
\pgfpathmoveto{\pgfqpoint{0.000000in}{0.000000in}}%
\pgfpathlineto{\pgfqpoint{0.000000in}{0.055556in}}%
\pgfusepath{stroke,fill}%
}%
\begin{pgfscope}%
\pgfsys@transformshift{1.893118in}{0.489833in}%
\pgfsys@useobject{currentmarker}{}%
\end{pgfscope}%
\end{pgfscope}%
\begin{pgfscope}%
\pgfsetbuttcap%
\pgfsetroundjoin%
\definecolor{currentfill}{rgb}{0.000000,0.000000,0.000000}%
\pgfsetfillcolor{currentfill}%
\pgfsetlinewidth{0.501875pt}%
\definecolor{currentstroke}{rgb}{0.000000,0.000000,0.000000}%
\pgfsetstrokecolor{currentstroke}%
\pgfsetdash{}{0pt}%
\pgfsys@defobject{currentmarker}{\pgfqpoint{0.000000in}{-0.055556in}}{\pgfqpoint{0.000000in}{0.000000in}}{%
\pgfpathmoveto{\pgfqpoint{0.000000in}{0.000000in}}%
\pgfpathlineto{\pgfqpoint{0.000000in}{-0.055556in}}%
\pgfusepath{stroke,fill}%
}%
\begin{pgfscope}%
\pgfsys@transformshift{1.893118in}{2.148139in}%
\pgfsys@useobject{currentmarker}{}%
\end{pgfscope}%
\end{pgfscope}%
\begin{pgfscope}%
\pgftext[x=1.893118in,y=0.434277in,,top]{{\rmfamily\fontsize{11.000000}{13.200000}\selectfont 2.2}}%
\end{pgfscope}%
\begin{pgfscope}%
\pgfsetbuttcap%
\pgfsetroundjoin%
\definecolor{currentfill}{rgb}{0.000000,0.000000,0.000000}%
\pgfsetfillcolor{currentfill}%
\pgfsetlinewidth{0.501875pt}%
\definecolor{currentstroke}{rgb}{0.000000,0.000000,0.000000}%
\pgfsetstrokecolor{currentstroke}%
\pgfsetdash{}{0pt}%
\pgfsys@defobject{currentmarker}{\pgfqpoint{0.000000in}{0.000000in}}{\pgfqpoint{0.000000in}{0.055556in}}{%
\pgfpathmoveto{\pgfqpoint{0.000000in}{0.000000in}}%
\pgfpathlineto{\pgfqpoint{0.000000in}{0.055556in}}%
\pgfusepath{stroke,fill}%
}%
\begin{pgfscope}%
\pgfsys@transformshift{2.508746in}{0.489833in}%
\pgfsys@useobject{currentmarker}{}%
\end{pgfscope}%
\end{pgfscope}%
\begin{pgfscope}%
\pgfsetbuttcap%
\pgfsetroundjoin%
\definecolor{currentfill}{rgb}{0.000000,0.000000,0.000000}%
\pgfsetfillcolor{currentfill}%
\pgfsetlinewidth{0.501875pt}%
\definecolor{currentstroke}{rgb}{0.000000,0.000000,0.000000}%
\pgfsetstrokecolor{currentstroke}%
\pgfsetdash{}{0pt}%
\pgfsys@defobject{currentmarker}{\pgfqpoint{0.000000in}{-0.055556in}}{\pgfqpoint{0.000000in}{0.000000in}}{%
\pgfpathmoveto{\pgfqpoint{0.000000in}{0.000000in}}%
\pgfpathlineto{\pgfqpoint{0.000000in}{-0.055556in}}%
\pgfusepath{stroke,fill}%
}%
\begin{pgfscope}%
\pgfsys@transformshift{2.508746in}{2.148139in}%
\pgfsys@useobject{currentmarker}{}%
\end{pgfscope}%
\end{pgfscope}%
\begin{pgfscope}%
\pgftext[x=2.508746in,y=0.434277in,,top]{{\rmfamily\fontsize{11.000000}{13.200000}\selectfont 2.3}}%
\end{pgfscope}%
\begin{pgfscope}%
\pgfsetbuttcap%
\pgfsetroundjoin%
\definecolor{currentfill}{rgb}{0.000000,0.000000,0.000000}%
\pgfsetfillcolor{currentfill}%
\pgfsetlinewidth{0.501875pt}%
\definecolor{currentstroke}{rgb}{0.000000,0.000000,0.000000}%
\pgfsetstrokecolor{currentstroke}%
\pgfsetdash{}{0pt}%
\pgfsys@defobject{currentmarker}{\pgfqpoint{0.000000in}{0.000000in}}{\pgfqpoint{0.000000in}{0.055556in}}{%
\pgfpathmoveto{\pgfqpoint{0.000000in}{0.000000in}}%
\pgfpathlineto{\pgfqpoint{0.000000in}{0.055556in}}%
\pgfusepath{stroke,fill}%
}%
\begin{pgfscope}%
\pgfsys@transformshift{3.124375in}{0.489833in}%
\pgfsys@useobject{currentmarker}{}%
\end{pgfscope}%
\end{pgfscope}%
\begin{pgfscope}%
\pgfsetbuttcap%
\pgfsetroundjoin%
\definecolor{currentfill}{rgb}{0.000000,0.000000,0.000000}%
\pgfsetfillcolor{currentfill}%
\pgfsetlinewidth{0.501875pt}%
\definecolor{currentstroke}{rgb}{0.000000,0.000000,0.000000}%
\pgfsetstrokecolor{currentstroke}%
\pgfsetdash{}{0pt}%
\pgfsys@defobject{currentmarker}{\pgfqpoint{0.000000in}{-0.055556in}}{\pgfqpoint{0.000000in}{0.000000in}}{%
\pgfpathmoveto{\pgfqpoint{0.000000in}{0.000000in}}%
\pgfpathlineto{\pgfqpoint{0.000000in}{-0.055556in}}%
\pgfusepath{stroke,fill}%
}%
\begin{pgfscope}%
\pgfsys@transformshift{3.124375in}{2.148139in}%
\pgfsys@useobject{currentmarker}{}%
\end{pgfscope}%
\end{pgfscope}%
\begin{pgfscope}%
\pgftext[x=3.124375in,y=0.434277in,,top]{{\rmfamily\fontsize{11.000000}{13.200000}\selectfont 2.4}}%
\end{pgfscope}%
\begin{pgfscope}%
\pgftext[x=1.893118in,y=0.229166in,,top]{{\rmfamily\fontsize{11.000000}{13.200000}\selectfont \(\displaystyle k_bT\)   [ \(\displaystyle J\) ]}}%
\end{pgfscope}%
\begin{pgfscope}%
\pgfsetbuttcap%
\pgfsetroundjoin%
\definecolor{currentfill}{rgb}{0.000000,0.000000,0.000000}%
\pgfsetfillcolor{currentfill}%
\pgfsetlinewidth{0.501875pt}%
\definecolor{currentstroke}{rgb}{0.000000,0.000000,0.000000}%
\pgfsetstrokecolor{currentstroke}%
\pgfsetdash{}{0pt}%
\pgfsys@defobject{currentmarker}{\pgfqpoint{0.000000in}{0.000000in}}{\pgfqpoint{0.055556in}{0.000000in}}{%
\pgfpathmoveto{\pgfqpoint{0.000000in}{0.000000in}}%
\pgfpathlineto{\pgfqpoint{0.055556in}{0.000000in}}%
\pgfusepath{stroke,fill}%
}%
\begin{pgfscope}%
\pgfsys@transformshift{0.661861in}{0.489833in}%
\pgfsys@useobject{currentmarker}{}%
\end{pgfscope}%
\end{pgfscope}%
\begin{pgfscope}%
\pgfsetbuttcap%
\pgfsetroundjoin%
\definecolor{currentfill}{rgb}{0.000000,0.000000,0.000000}%
\pgfsetfillcolor{currentfill}%
\pgfsetlinewidth{0.501875pt}%
\definecolor{currentstroke}{rgb}{0.000000,0.000000,0.000000}%
\pgfsetstrokecolor{currentstroke}%
\pgfsetdash{}{0pt}%
\pgfsys@defobject{currentmarker}{\pgfqpoint{-0.055556in}{0.000000in}}{\pgfqpoint{0.000000in}{0.000000in}}{%
\pgfpathmoveto{\pgfqpoint{0.000000in}{0.000000in}}%
\pgfpathlineto{\pgfqpoint{-0.055556in}{0.000000in}}%
\pgfusepath{stroke,fill}%
}%
\begin{pgfscope}%
\pgfsys@transformshift{3.124375in}{0.489833in}%
\pgfsys@useobject{currentmarker}{}%
\end{pgfscope}%
\end{pgfscope}%
\begin{pgfscope}%
\pgftext[x=0.606305in,y=0.489833in,right,]{{\rmfamily\fontsize{11.000000}{13.200000}\selectfont −1.8}}%
\end{pgfscope}%
\begin{pgfscope}%
\pgfsetbuttcap%
\pgfsetroundjoin%
\definecolor{currentfill}{rgb}{0.000000,0.000000,0.000000}%
\pgfsetfillcolor{currentfill}%
\pgfsetlinewidth{0.501875pt}%
\definecolor{currentstroke}{rgb}{0.000000,0.000000,0.000000}%
\pgfsetstrokecolor{currentstroke}%
\pgfsetdash{}{0pt}%
\pgfsys@defobject{currentmarker}{\pgfqpoint{0.000000in}{0.000000in}}{\pgfqpoint{0.055556in}{0.000000in}}{%
\pgfpathmoveto{\pgfqpoint{0.000000in}{0.000000in}}%
\pgfpathlineto{\pgfqpoint{0.055556in}{0.000000in}}%
\pgfusepath{stroke,fill}%
}%
\begin{pgfscope}%
\pgfsys@transformshift{0.661861in}{0.766217in}%
\pgfsys@useobject{currentmarker}{}%
\end{pgfscope}%
\end{pgfscope}%
\begin{pgfscope}%
\pgfsetbuttcap%
\pgfsetroundjoin%
\definecolor{currentfill}{rgb}{0.000000,0.000000,0.000000}%
\pgfsetfillcolor{currentfill}%
\pgfsetlinewidth{0.501875pt}%
\definecolor{currentstroke}{rgb}{0.000000,0.000000,0.000000}%
\pgfsetstrokecolor{currentstroke}%
\pgfsetdash{}{0pt}%
\pgfsys@defobject{currentmarker}{\pgfqpoint{-0.055556in}{0.000000in}}{\pgfqpoint{0.000000in}{0.000000in}}{%
\pgfpathmoveto{\pgfqpoint{0.000000in}{0.000000in}}%
\pgfpathlineto{\pgfqpoint{-0.055556in}{0.000000in}}%
\pgfusepath{stroke,fill}%
}%
\begin{pgfscope}%
\pgfsys@transformshift{3.124375in}{0.766217in}%
\pgfsys@useobject{currentmarker}{}%
\end{pgfscope}%
\end{pgfscope}%
\begin{pgfscope}%
\pgftext[x=0.606305in,y=0.766217in,right,]{{\rmfamily\fontsize{11.000000}{13.200000}\selectfont −1.7}}%
\end{pgfscope}%
\begin{pgfscope}%
\pgfsetbuttcap%
\pgfsetroundjoin%
\definecolor{currentfill}{rgb}{0.000000,0.000000,0.000000}%
\pgfsetfillcolor{currentfill}%
\pgfsetlinewidth{0.501875pt}%
\definecolor{currentstroke}{rgb}{0.000000,0.000000,0.000000}%
\pgfsetstrokecolor{currentstroke}%
\pgfsetdash{}{0pt}%
\pgfsys@defobject{currentmarker}{\pgfqpoint{0.000000in}{0.000000in}}{\pgfqpoint{0.055556in}{0.000000in}}{%
\pgfpathmoveto{\pgfqpoint{0.000000in}{0.000000in}}%
\pgfpathlineto{\pgfqpoint{0.055556in}{0.000000in}}%
\pgfusepath{stroke,fill}%
}%
\begin{pgfscope}%
\pgfsys@transformshift{0.661861in}{1.042602in}%
\pgfsys@useobject{currentmarker}{}%
\end{pgfscope}%
\end{pgfscope}%
\begin{pgfscope}%
\pgfsetbuttcap%
\pgfsetroundjoin%
\definecolor{currentfill}{rgb}{0.000000,0.000000,0.000000}%
\pgfsetfillcolor{currentfill}%
\pgfsetlinewidth{0.501875pt}%
\definecolor{currentstroke}{rgb}{0.000000,0.000000,0.000000}%
\pgfsetstrokecolor{currentstroke}%
\pgfsetdash{}{0pt}%
\pgfsys@defobject{currentmarker}{\pgfqpoint{-0.055556in}{0.000000in}}{\pgfqpoint{0.000000in}{0.000000in}}{%
\pgfpathmoveto{\pgfqpoint{0.000000in}{0.000000in}}%
\pgfpathlineto{\pgfqpoint{-0.055556in}{0.000000in}}%
\pgfusepath{stroke,fill}%
}%
\begin{pgfscope}%
\pgfsys@transformshift{3.124375in}{1.042602in}%
\pgfsys@useobject{currentmarker}{}%
\end{pgfscope}%
\end{pgfscope}%
\begin{pgfscope}%
\pgftext[x=0.606305in,y=1.042602in,right,]{{\rmfamily\fontsize{11.000000}{13.200000}\selectfont −1.6}}%
\end{pgfscope}%
\begin{pgfscope}%
\pgfsetbuttcap%
\pgfsetroundjoin%
\definecolor{currentfill}{rgb}{0.000000,0.000000,0.000000}%
\pgfsetfillcolor{currentfill}%
\pgfsetlinewidth{0.501875pt}%
\definecolor{currentstroke}{rgb}{0.000000,0.000000,0.000000}%
\pgfsetstrokecolor{currentstroke}%
\pgfsetdash{}{0pt}%
\pgfsys@defobject{currentmarker}{\pgfqpoint{0.000000in}{0.000000in}}{\pgfqpoint{0.055556in}{0.000000in}}{%
\pgfpathmoveto{\pgfqpoint{0.000000in}{0.000000in}}%
\pgfpathlineto{\pgfqpoint{0.055556in}{0.000000in}}%
\pgfusepath{stroke,fill}%
}%
\begin{pgfscope}%
\pgfsys@transformshift{0.661861in}{1.318986in}%
\pgfsys@useobject{currentmarker}{}%
\end{pgfscope}%
\end{pgfscope}%
\begin{pgfscope}%
\pgfsetbuttcap%
\pgfsetroundjoin%
\definecolor{currentfill}{rgb}{0.000000,0.000000,0.000000}%
\pgfsetfillcolor{currentfill}%
\pgfsetlinewidth{0.501875pt}%
\definecolor{currentstroke}{rgb}{0.000000,0.000000,0.000000}%
\pgfsetstrokecolor{currentstroke}%
\pgfsetdash{}{0pt}%
\pgfsys@defobject{currentmarker}{\pgfqpoint{-0.055556in}{0.000000in}}{\pgfqpoint{0.000000in}{0.000000in}}{%
\pgfpathmoveto{\pgfqpoint{0.000000in}{0.000000in}}%
\pgfpathlineto{\pgfqpoint{-0.055556in}{0.000000in}}%
\pgfusepath{stroke,fill}%
}%
\begin{pgfscope}%
\pgfsys@transformshift{3.124375in}{1.318986in}%
\pgfsys@useobject{currentmarker}{}%
\end{pgfscope}%
\end{pgfscope}%
\begin{pgfscope}%
\pgftext[x=0.606305in,y=1.318986in,right,]{{\rmfamily\fontsize{11.000000}{13.200000}\selectfont −1.5}}%
\end{pgfscope}%
\begin{pgfscope}%
\pgfsetbuttcap%
\pgfsetroundjoin%
\definecolor{currentfill}{rgb}{0.000000,0.000000,0.000000}%
\pgfsetfillcolor{currentfill}%
\pgfsetlinewidth{0.501875pt}%
\definecolor{currentstroke}{rgb}{0.000000,0.000000,0.000000}%
\pgfsetstrokecolor{currentstroke}%
\pgfsetdash{}{0pt}%
\pgfsys@defobject{currentmarker}{\pgfqpoint{0.000000in}{0.000000in}}{\pgfqpoint{0.055556in}{0.000000in}}{%
\pgfpathmoveto{\pgfqpoint{0.000000in}{0.000000in}}%
\pgfpathlineto{\pgfqpoint{0.055556in}{0.000000in}}%
\pgfusepath{stroke,fill}%
}%
\begin{pgfscope}%
\pgfsys@transformshift{0.661861in}{1.595370in}%
\pgfsys@useobject{currentmarker}{}%
\end{pgfscope}%
\end{pgfscope}%
\begin{pgfscope}%
\pgfsetbuttcap%
\pgfsetroundjoin%
\definecolor{currentfill}{rgb}{0.000000,0.000000,0.000000}%
\pgfsetfillcolor{currentfill}%
\pgfsetlinewidth{0.501875pt}%
\definecolor{currentstroke}{rgb}{0.000000,0.000000,0.000000}%
\pgfsetstrokecolor{currentstroke}%
\pgfsetdash{}{0pt}%
\pgfsys@defobject{currentmarker}{\pgfqpoint{-0.055556in}{0.000000in}}{\pgfqpoint{0.000000in}{0.000000in}}{%
\pgfpathmoveto{\pgfqpoint{0.000000in}{0.000000in}}%
\pgfpathlineto{\pgfqpoint{-0.055556in}{0.000000in}}%
\pgfusepath{stroke,fill}%
}%
\begin{pgfscope}%
\pgfsys@transformshift{3.124375in}{1.595370in}%
\pgfsys@useobject{currentmarker}{}%
\end{pgfscope}%
\end{pgfscope}%
\begin{pgfscope}%
\pgftext[x=0.606305in,y=1.595370in,right,]{{\rmfamily\fontsize{11.000000}{13.200000}\selectfont −1.4}}%
\end{pgfscope}%
\begin{pgfscope}%
\pgfsetbuttcap%
\pgfsetroundjoin%
\definecolor{currentfill}{rgb}{0.000000,0.000000,0.000000}%
\pgfsetfillcolor{currentfill}%
\pgfsetlinewidth{0.501875pt}%
\definecolor{currentstroke}{rgb}{0.000000,0.000000,0.000000}%
\pgfsetstrokecolor{currentstroke}%
\pgfsetdash{}{0pt}%
\pgfsys@defobject{currentmarker}{\pgfqpoint{0.000000in}{0.000000in}}{\pgfqpoint{0.055556in}{0.000000in}}{%
\pgfpathmoveto{\pgfqpoint{0.000000in}{0.000000in}}%
\pgfpathlineto{\pgfqpoint{0.055556in}{0.000000in}}%
\pgfusepath{stroke,fill}%
}%
\begin{pgfscope}%
\pgfsys@transformshift{0.661861in}{1.871755in}%
\pgfsys@useobject{currentmarker}{}%
\end{pgfscope}%
\end{pgfscope}%
\begin{pgfscope}%
\pgfsetbuttcap%
\pgfsetroundjoin%
\definecolor{currentfill}{rgb}{0.000000,0.000000,0.000000}%
\pgfsetfillcolor{currentfill}%
\pgfsetlinewidth{0.501875pt}%
\definecolor{currentstroke}{rgb}{0.000000,0.000000,0.000000}%
\pgfsetstrokecolor{currentstroke}%
\pgfsetdash{}{0pt}%
\pgfsys@defobject{currentmarker}{\pgfqpoint{-0.055556in}{0.000000in}}{\pgfqpoint{0.000000in}{0.000000in}}{%
\pgfpathmoveto{\pgfqpoint{0.000000in}{0.000000in}}%
\pgfpathlineto{\pgfqpoint{-0.055556in}{0.000000in}}%
\pgfusepath{stroke,fill}%
}%
\begin{pgfscope}%
\pgfsys@transformshift{3.124375in}{1.871755in}%
\pgfsys@useobject{currentmarker}{}%
\end{pgfscope}%
\end{pgfscope}%
\begin{pgfscope}%
\pgftext[x=0.606305in,y=1.871755in,right,]{{\rmfamily\fontsize{11.000000}{13.200000}\selectfont −1.3}}%
\end{pgfscope}%
\begin{pgfscope}%
\pgfsetbuttcap%
\pgfsetroundjoin%
\definecolor{currentfill}{rgb}{0.000000,0.000000,0.000000}%
\pgfsetfillcolor{currentfill}%
\pgfsetlinewidth{0.501875pt}%
\definecolor{currentstroke}{rgb}{0.000000,0.000000,0.000000}%
\pgfsetstrokecolor{currentstroke}%
\pgfsetdash{}{0pt}%
\pgfsys@defobject{currentmarker}{\pgfqpoint{0.000000in}{0.000000in}}{\pgfqpoint{0.055556in}{0.000000in}}{%
\pgfpathmoveto{\pgfqpoint{0.000000in}{0.000000in}}%
\pgfpathlineto{\pgfqpoint{0.055556in}{0.000000in}}%
\pgfusepath{stroke,fill}%
}%
\begin{pgfscope}%
\pgfsys@transformshift{0.661861in}{2.148139in}%
\pgfsys@useobject{currentmarker}{}%
\end{pgfscope}%
\end{pgfscope}%
\begin{pgfscope}%
\pgfsetbuttcap%
\pgfsetroundjoin%
\definecolor{currentfill}{rgb}{0.000000,0.000000,0.000000}%
\pgfsetfillcolor{currentfill}%
\pgfsetlinewidth{0.501875pt}%
\definecolor{currentstroke}{rgb}{0.000000,0.000000,0.000000}%
\pgfsetstrokecolor{currentstroke}%
\pgfsetdash{}{0pt}%
\pgfsys@defobject{currentmarker}{\pgfqpoint{-0.055556in}{0.000000in}}{\pgfqpoint{0.000000in}{0.000000in}}{%
\pgfpathmoveto{\pgfqpoint{0.000000in}{0.000000in}}%
\pgfpathlineto{\pgfqpoint{-0.055556in}{0.000000in}}%
\pgfusepath{stroke,fill}%
}%
\begin{pgfscope}%
\pgfsys@transformshift{3.124375in}{2.148139in}%
\pgfsys@useobject{currentmarker}{}%
\end{pgfscope}%
\end{pgfscope}%
\begin{pgfscope}%
\pgftext[x=0.606305in,y=2.148139in,right,]{{\rmfamily\fontsize{11.000000}{13.200000}\selectfont −1.2}}%
\end{pgfscope}%
\begin{pgfscope}%
\pgftext[x=0.229166in,y=1.318986in,,bottom,rotate=90.000000]{{\rmfamily\fontsize{11.000000}{13.200000}\selectfont \(\displaystyle \langle E \rangle\)   [ \(\displaystyle \cdot\) ]}}%
\end{pgfscope}%
\begin{pgfscope}%
\pgftext[x=1.893118in,y=2.217583in,,base]{{\rmfamily\fontsize{11.000000}{13.200000}\selectfont Energy vs thermal energy}}%
\end{pgfscope}%
\begin{pgfscope}%
\pgfsetbuttcap%
\pgfsetmiterjoin%
\definecolor{currentfill}{rgb}{1.000000,1.000000,1.000000}%
\pgfsetfillcolor{currentfill}%
\pgfsetlinewidth{0.000000pt}%
\definecolor{currentstroke}{rgb}{0.000000,0.000000,0.000000}%
\pgfsetstrokecolor{currentstroke}%
\pgfsetstrokeopacity{0.000000}%
\pgfsetdash{}{0pt}%
\pgfpathmoveto{\pgfqpoint{3.765458in}{0.489833in}}%
\pgfpathlineto{\pgfqpoint{6.227972in}{0.489833in}}%
\pgfpathlineto{\pgfqpoint{6.227972in}{2.148139in}}%
\pgfpathlineto{\pgfqpoint{3.765458in}{2.148139in}}%
\pgfpathclose%
\pgfusepath{fill}%
\end{pgfscope}%
\begin{pgfscope}%
\pgfpathrectangle{\pgfqpoint{3.765458in}{0.489833in}}{\pgfqpoint{2.462514in}{1.658306in}} %
\pgfusepath{clip}%
\pgfsetbuttcap%
\pgfsetroundjoin%
\pgfsetlinewidth{1.003750pt}%
\definecolor{currentstroke}{rgb}{0.000000,0.000000,0.000000}%
\pgfsetstrokecolor{currentstroke}%
\pgfsetdash{{8.000000pt}{4.000000pt}{2.000000pt}{4.000000pt}{2.000000pt}{4.000000pt}}{0.000000pt}%
\pgfpathmoveto{\pgfqpoint{3.765458in}{2.001506in}}%
\pgfpathlineto{\pgfqpoint{4.048647in}{1.972678in}}%
\pgfpathlineto{\pgfqpoint{4.196398in}{1.955517in}}%
\pgfpathlineto{\pgfqpoint{4.331836in}{1.937578in}}%
\pgfpathlineto{\pgfqpoint{4.454962in}{1.918995in}}%
\pgfpathlineto{\pgfqpoint{4.565775in}{1.900094in}}%
\pgfpathlineto{\pgfqpoint{4.664276in}{1.881279in}}%
\pgfpathlineto{\pgfqpoint{4.762776in}{1.860251in}}%
\pgfpathlineto{\pgfqpoint{4.848964in}{1.839791in}}%
\pgfpathlineto{\pgfqpoint{4.935152in}{1.817187in}}%
\pgfpathlineto{\pgfqpoint{5.021340in}{1.792188in}}%
\pgfpathlineto{\pgfqpoint{5.095216in}{1.768644in}}%
\pgfpathlineto{\pgfqpoint{5.169091in}{1.742991in}}%
\pgfpathlineto{\pgfqpoint{5.255279in}{1.710606in}}%
\pgfpathlineto{\pgfqpoint{5.341467in}{1.675885in}}%
\pgfpathlineto{\pgfqpoint{5.415343in}{1.644076in}}%
\pgfpathlineto{\pgfqpoint{5.501531in}{1.604284in}}%
\pgfpathlineto{\pgfqpoint{5.612344in}{1.549959in}}%
\pgfpathlineto{\pgfqpoint{5.784720in}{1.462371in}}%
\pgfpathlineto{\pgfqpoint{6.080221in}{1.311518in}}%
\pgfpathlineto{\pgfqpoint{6.227972in}{1.239089in}}%
\pgfpathlineto{\pgfqpoint{6.227972in}{1.239089in}}%
\pgfusepath{stroke}%
\end{pgfscope}%
\begin{pgfscope}%
\pgfpathrectangle{\pgfqpoint{3.765458in}{0.489833in}}{\pgfqpoint{2.462514in}{1.658306in}} %
\pgfusepath{clip}%
\pgfsetbuttcap%
\pgfsetroundjoin%
\pgfsetlinewidth{1.003750pt}%
\definecolor{currentstroke}{rgb}{0.000000,0.000000,0.000000}%
\pgfsetstrokecolor{currentstroke}%
\pgfsetdash{{6.000000pt}{6.000000pt}}{0.000000pt}%
\pgfpathmoveto{\pgfqpoint{3.765458in}{2.001439in}}%
\pgfpathlineto{\pgfqpoint{4.036335in}{1.974311in}}%
\pgfpathlineto{\pgfqpoint{4.184085in}{1.957351in}}%
\pgfpathlineto{\pgfqpoint{4.319524in}{1.939508in}}%
\pgfpathlineto{\pgfqpoint{4.442649in}{1.921003in}}%
\pgfpathlineto{\pgfqpoint{4.553463in}{1.902221in}}%
\pgfpathlineto{\pgfqpoint{4.651963in}{1.883453in}}%
\pgfpathlineto{\pgfqpoint{4.738151in}{1.864893in}}%
\pgfpathlineto{\pgfqpoint{4.812027in}{1.847025in}}%
\pgfpathlineto{\pgfqpoint{4.885902in}{1.827038in}}%
\pgfpathlineto{\pgfqpoint{4.947465in}{1.808387in}}%
\pgfpathlineto{\pgfqpoint{4.996715in}{1.791580in}}%
\pgfpathlineto{\pgfqpoint{5.045965in}{1.772579in}}%
\pgfpathlineto{\pgfqpoint{5.095216in}{1.750906in}}%
\pgfpathlineto{\pgfqpoint{5.132153in}{1.732609in}}%
\pgfpathlineto{\pgfqpoint{5.169091in}{1.712328in}}%
\pgfpathlineto{\pgfqpoint{5.206029in}{1.689842in}}%
\pgfpathlineto{\pgfqpoint{5.242967in}{1.664927in}}%
\pgfpathlineto{\pgfqpoint{5.279904in}{1.637217in}}%
\pgfpathlineto{\pgfqpoint{5.316842in}{1.605771in}}%
\pgfpathlineto{\pgfqpoint{5.341467in}{1.582197in}}%
\pgfpathlineto{\pgfqpoint{5.378405in}{1.542784in}}%
\pgfpathlineto{\pgfqpoint{5.415343in}{1.499439in}}%
\pgfpathlineto{\pgfqpoint{5.464593in}{1.437202in}}%
\pgfpathlineto{\pgfqpoint{5.513843in}{1.370395in}}%
\pgfpathlineto{\pgfqpoint{5.636969in}{1.197112in}}%
\pgfpathlineto{\pgfqpoint{5.686219in}{1.134387in}}%
\pgfpathlineto{\pgfqpoint{5.723157in}{1.090088in}}%
\pgfpathlineto{\pgfqpoint{5.760095in}{1.048807in}}%
\pgfpathlineto{\pgfqpoint{5.797032in}{1.010823in}}%
\pgfpathlineto{\pgfqpoint{5.833970in}{0.976465in}}%
\pgfpathlineto{\pgfqpoint{5.870908in}{0.946285in}}%
\pgfpathlineto{\pgfqpoint{5.907845in}{0.919878in}}%
\pgfpathlineto{\pgfqpoint{5.957096in}{0.888363in}}%
\pgfpathlineto{\pgfqpoint{6.006346in}{0.860252in}}%
\pgfpathlineto{\pgfqpoint{6.055596in}{0.835407in}}%
\pgfpathlineto{\pgfqpoint{6.104847in}{0.813687in}}%
\pgfpathlineto{\pgfqpoint{6.154097in}{0.794897in}}%
\pgfpathlineto{\pgfqpoint{6.227972in}{0.770037in}}%
\pgfpathlineto{\pgfqpoint{6.227972in}{0.770037in}}%
\pgfusepath{stroke}%
\end{pgfscope}%
\begin{pgfscope}%
\pgfpathrectangle{\pgfqpoint{3.765458in}{0.489833in}}{\pgfqpoint{2.462514in}{1.658306in}} %
\pgfusepath{clip}%
\pgfsetbuttcap%
\pgfsetroundjoin%
\pgfsetlinewidth{1.003750pt}%
\definecolor{currentstroke}{rgb}{0.000000,0.000000,0.000000}%
\pgfsetstrokecolor{currentstroke}%
\pgfsetdash{{2.000000pt}{2.000000pt}}{0.000000pt}%
\pgfpathmoveto{\pgfqpoint{3.765458in}{2.001412in}}%
\pgfpathlineto{\pgfqpoint{4.036335in}{1.974326in}}%
\pgfpathlineto{\pgfqpoint{4.184085in}{1.957377in}}%
\pgfpathlineto{\pgfqpoint{4.319524in}{1.939530in}}%
\pgfpathlineto{\pgfqpoint{4.442649in}{1.921005in}}%
\pgfpathlineto{\pgfqpoint{4.553463in}{1.902189in}}%
\pgfpathlineto{\pgfqpoint{4.651963in}{1.883379in}}%
\pgfpathlineto{\pgfqpoint{4.738151in}{1.864782in}}%
\pgfpathlineto{\pgfqpoint{4.824339in}{1.843708in}}%
\pgfpathlineto{\pgfqpoint{4.898215in}{1.823322in}}%
\pgfpathlineto{\pgfqpoint{4.959777in}{1.804224in}}%
\pgfpathlineto{\pgfqpoint{5.009028in}{1.786975in}}%
\pgfpathlineto{\pgfqpoint{5.058278in}{1.767484in}}%
\pgfpathlineto{\pgfqpoint{5.107528in}{1.745297in}}%
\pgfpathlineto{\pgfqpoint{5.144466in}{1.726581in}}%
\pgfpathlineto{\pgfqpoint{5.181404in}{1.705537in}}%
\pgfpathlineto{\pgfqpoint{5.218341in}{1.681479in}}%
\pgfpathlineto{\pgfqpoint{5.255279in}{1.653706in}}%
\pgfpathlineto{\pgfqpoint{5.279904in}{1.632491in}}%
\pgfpathlineto{\pgfqpoint{5.304529in}{1.608252in}}%
\pgfpathlineto{\pgfqpoint{5.329155in}{1.580204in}}%
\pgfpathlineto{\pgfqpoint{5.353780in}{1.547599in}}%
\pgfpathlineto{\pgfqpoint{5.378405in}{1.509949in}}%
\pgfpathlineto{\pgfqpoint{5.403030in}{1.466923in}}%
\pgfpathlineto{\pgfqpoint{5.427655in}{1.419072in}}%
\pgfpathlineto{\pgfqpoint{5.464593in}{1.340973in}}%
\pgfpathlineto{\pgfqpoint{5.550781in}{1.153394in}}%
\pgfpathlineto{\pgfqpoint{5.587719in}{1.080527in}}%
\pgfpathlineto{\pgfqpoint{5.624656in}{1.013999in}}%
\pgfpathlineto{\pgfqpoint{5.649281in}{0.973663in}}%
\pgfpathlineto{\pgfqpoint{5.673907in}{0.937273in}}%
\pgfpathlineto{\pgfqpoint{5.698532in}{0.905007in}}%
\pgfpathlineto{\pgfqpoint{5.723157in}{0.876638in}}%
\pgfpathlineto{\pgfqpoint{5.747782in}{0.851874in}}%
\pgfpathlineto{\pgfqpoint{5.772407in}{0.830267in}}%
\pgfpathlineto{\pgfqpoint{5.797032in}{0.811343in}}%
\pgfpathlineto{\pgfqpoint{5.821657in}{0.794601in}}%
\pgfpathlineto{\pgfqpoint{5.858595in}{0.772396in}}%
\pgfpathlineto{\pgfqpoint{5.895533in}{0.752646in}}%
\pgfpathlineto{\pgfqpoint{5.932471in}{0.735613in}}%
\pgfpathlineto{\pgfqpoint{5.969408in}{0.721299in}}%
\pgfpathlineto{\pgfqpoint{6.018659in}{0.705360in}}%
\pgfpathlineto{\pgfqpoint{6.080221in}{0.688529in}}%
\pgfpathlineto{\pgfqpoint{6.141784in}{0.674242in}}%
\pgfpathlineto{\pgfqpoint{6.227972in}{0.657029in}}%
\pgfpathlineto{\pgfqpoint{6.227972in}{0.657029in}}%
\pgfusepath{stroke}%
\end{pgfscope}%
\begin{pgfscope}%
\pgfpathrectangle{\pgfqpoint{3.765458in}{0.489833in}}{\pgfqpoint{2.462514in}{1.658306in}} %
\pgfusepath{clip}%
\pgfsetrectcap%
\pgfsetroundjoin%
\pgfsetlinewidth{1.003750pt}%
\definecolor{currentstroke}{rgb}{0.000000,0.000000,0.000000}%
\pgfsetstrokecolor{currentstroke}%
\pgfsetdash{}{0pt}%
\pgfpathmoveto{\pgfqpoint{3.765458in}{2.001356in}}%
\pgfpathlineto{\pgfqpoint{4.036335in}{1.974330in}}%
\pgfpathlineto{\pgfqpoint{4.184085in}{1.957371in}}%
\pgfpathlineto{\pgfqpoint{4.319524in}{1.939491in}}%
\pgfpathlineto{\pgfqpoint{4.442649in}{1.920963in}}%
\pgfpathlineto{\pgfqpoint{4.553463in}{1.902201in}}%
\pgfpathlineto{\pgfqpoint{4.651963in}{1.883481in}}%
\pgfpathlineto{\pgfqpoint{4.738151in}{1.864941in}}%
\pgfpathlineto{\pgfqpoint{4.812027in}{1.847043in}}%
\pgfpathlineto{\pgfqpoint{4.885902in}{1.826955in}}%
\pgfpathlineto{\pgfqpoint{4.947465in}{1.808180in}}%
\pgfpathlineto{\pgfqpoint{5.009028in}{1.786811in}}%
\pgfpathlineto{\pgfqpoint{5.058278in}{1.767312in}}%
\pgfpathlineto{\pgfqpoint{5.107528in}{1.745237in}}%
\pgfpathlineto{\pgfqpoint{5.144466in}{1.726687in}}%
\pgfpathlineto{\pgfqpoint{5.181404in}{1.705659in}}%
\pgfpathlineto{\pgfqpoint{5.206029in}{1.689758in}}%
\pgfpathlineto{\pgfqpoint{5.230654in}{1.671996in}}%
\pgfpathlineto{\pgfqpoint{5.255279in}{1.652063in}}%
\pgfpathlineto{\pgfqpoint{5.279904in}{1.629492in}}%
\pgfpathlineto{\pgfqpoint{5.304529in}{1.603655in}}%
\pgfpathlineto{\pgfqpoint{5.329155in}{1.573918in}}%
\pgfpathlineto{\pgfqpoint{5.353780in}{1.539402in}}%
\pgfpathlineto{\pgfqpoint{5.366092in}{1.519632in}}%
\pgfpathlineto{\pgfqpoint{5.378405in}{1.497681in}}%
\pgfpathlineto{\pgfqpoint{5.390717in}{1.473173in}}%
\pgfpathlineto{\pgfqpoint{5.415343in}{1.416280in}}%
\pgfpathlineto{\pgfqpoint{5.439968in}{1.351096in}}%
\pgfpathlineto{\pgfqpoint{5.476905in}{1.243825in}}%
\pgfpathlineto{\pgfqpoint{5.526156in}{1.097078in}}%
\pgfpathlineto{\pgfqpoint{5.550781in}{1.030045in}}%
\pgfpathlineto{\pgfqpoint{5.575406in}{0.971851in}}%
\pgfpathlineto{\pgfqpoint{5.600031in}{0.922418in}}%
\pgfpathlineto{\pgfqpoint{5.624656in}{0.880843in}}%
\pgfpathlineto{\pgfqpoint{5.649281in}{0.845579in}}%
\pgfpathlineto{\pgfqpoint{5.673907in}{0.814946in}}%
\pgfpathlineto{\pgfqpoint{5.698532in}{0.788256in}}%
\pgfpathlineto{\pgfqpoint{5.723157in}{0.765697in}}%
\pgfpathlineto{\pgfqpoint{5.747782in}{0.747207in}}%
\pgfpathlineto{\pgfqpoint{5.772407in}{0.731889in}}%
\pgfpathlineto{\pgfqpoint{5.809345in}{0.712637in}}%
\pgfpathlineto{\pgfqpoint{5.846283in}{0.696067in}}%
\pgfpathlineto{\pgfqpoint{5.895533in}{0.676706in}}%
\pgfpathlineto{\pgfqpoint{5.932471in}{0.664341in}}%
\pgfpathlineto{\pgfqpoint{5.969408in}{0.653919in}}%
\pgfpathlineto{\pgfqpoint{6.018659in}{0.642437in}}%
\pgfpathlineto{\pgfqpoint{6.080221in}{0.630749in}}%
\pgfpathlineto{\pgfqpoint{6.178722in}{0.614867in}}%
\pgfpathlineto{\pgfqpoint{6.227972in}{0.607284in}}%
\pgfpathlineto{\pgfqpoint{6.227972in}{0.607284in}}%
\pgfusepath{stroke}%
\end{pgfscope}%
\begin{pgfscope}%
\pgfsetrectcap%
\pgfsetmiterjoin%
\pgfsetlinewidth{1.003750pt}%
\definecolor{currentstroke}{rgb}{0.000000,0.000000,0.000000}%
\pgfsetstrokecolor{currentstroke}%
\pgfsetdash{}{0pt}%
\pgfpathmoveto{\pgfqpoint{3.765458in}{2.148139in}}%
\pgfpathlineto{\pgfqpoint{6.227972in}{2.148139in}}%
\pgfusepath{stroke}%
\end{pgfscope}%
\begin{pgfscope}%
\pgfsetrectcap%
\pgfsetmiterjoin%
\pgfsetlinewidth{1.003750pt}%
\definecolor{currentstroke}{rgb}{0.000000,0.000000,0.000000}%
\pgfsetstrokecolor{currentstroke}%
\pgfsetdash{}{0pt}%
\pgfpathmoveto{\pgfqpoint{6.227972in}{0.489833in}}%
\pgfpathlineto{\pgfqpoint{6.227972in}{2.148139in}}%
\pgfusepath{stroke}%
\end{pgfscope}%
\begin{pgfscope}%
\pgfsetrectcap%
\pgfsetmiterjoin%
\pgfsetlinewidth{1.003750pt}%
\definecolor{currentstroke}{rgb}{0.000000,0.000000,0.000000}%
\pgfsetstrokecolor{currentstroke}%
\pgfsetdash{}{0pt}%
\pgfpathmoveto{\pgfqpoint{3.765458in}{0.489833in}}%
\pgfpathlineto{\pgfqpoint{6.227972in}{0.489833in}}%
\pgfusepath{stroke}%
\end{pgfscope}%
\begin{pgfscope}%
\pgfsetrectcap%
\pgfsetmiterjoin%
\pgfsetlinewidth{1.003750pt}%
\definecolor{currentstroke}{rgb}{0.000000,0.000000,0.000000}%
\pgfsetstrokecolor{currentstroke}%
\pgfsetdash{}{0pt}%
\pgfpathmoveto{\pgfqpoint{3.765458in}{0.489833in}}%
\pgfpathlineto{\pgfqpoint{3.765458in}{2.148139in}}%
\pgfusepath{stroke}%
\end{pgfscope}%
\begin{pgfscope}%
\pgfsetbuttcap%
\pgfsetroundjoin%
\definecolor{currentfill}{rgb}{0.000000,0.000000,0.000000}%
\pgfsetfillcolor{currentfill}%
\pgfsetlinewidth{0.501875pt}%
\definecolor{currentstroke}{rgb}{0.000000,0.000000,0.000000}%
\pgfsetstrokecolor{currentstroke}%
\pgfsetdash{}{0pt}%
\pgfsys@defobject{currentmarker}{\pgfqpoint{0.000000in}{0.000000in}}{\pgfqpoint{0.000000in}{0.055556in}}{%
\pgfpathmoveto{\pgfqpoint{0.000000in}{0.000000in}}%
\pgfpathlineto{\pgfqpoint{0.000000in}{0.055556in}}%
\pgfusepath{stroke,fill}%
}%
\begin{pgfscope}%
\pgfsys@transformshift{3.765458in}{0.489833in}%
\pgfsys@useobject{currentmarker}{}%
\end{pgfscope}%
\end{pgfscope}%
\begin{pgfscope}%
\pgfsetbuttcap%
\pgfsetroundjoin%
\definecolor{currentfill}{rgb}{0.000000,0.000000,0.000000}%
\pgfsetfillcolor{currentfill}%
\pgfsetlinewidth{0.501875pt}%
\definecolor{currentstroke}{rgb}{0.000000,0.000000,0.000000}%
\pgfsetstrokecolor{currentstroke}%
\pgfsetdash{}{0pt}%
\pgfsys@defobject{currentmarker}{\pgfqpoint{0.000000in}{-0.055556in}}{\pgfqpoint{0.000000in}{0.000000in}}{%
\pgfpathmoveto{\pgfqpoint{0.000000in}{0.000000in}}%
\pgfpathlineto{\pgfqpoint{0.000000in}{-0.055556in}}%
\pgfusepath{stroke,fill}%
}%
\begin{pgfscope}%
\pgfsys@transformshift{3.765458in}{2.148139in}%
\pgfsys@useobject{currentmarker}{}%
\end{pgfscope}%
\end{pgfscope}%
\begin{pgfscope}%
\pgftext[x=3.765458in,y=0.434277in,,top]{{\rmfamily\fontsize{11.000000}{13.200000}\selectfont 2.0}}%
\end{pgfscope}%
\begin{pgfscope}%
\pgfsetbuttcap%
\pgfsetroundjoin%
\definecolor{currentfill}{rgb}{0.000000,0.000000,0.000000}%
\pgfsetfillcolor{currentfill}%
\pgfsetlinewidth{0.501875pt}%
\definecolor{currentstroke}{rgb}{0.000000,0.000000,0.000000}%
\pgfsetstrokecolor{currentstroke}%
\pgfsetdash{}{0pt}%
\pgfsys@defobject{currentmarker}{\pgfqpoint{0.000000in}{0.000000in}}{\pgfqpoint{0.000000in}{0.055556in}}{%
\pgfpathmoveto{\pgfqpoint{0.000000in}{0.000000in}}%
\pgfpathlineto{\pgfqpoint{0.000000in}{0.055556in}}%
\pgfusepath{stroke,fill}%
}%
\begin{pgfscope}%
\pgfsys@transformshift{4.381087in}{0.489833in}%
\pgfsys@useobject{currentmarker}{}%
\end{pgfscope}%
\end{pgfscope}%
\begin{pgfscope}%
\pgfsetbuttcap%
\pgfsetroundjoin%
\definecolor{currentfill}{rgb}{0.000000,0.000000,0.000000}%
\pgfsetfillcolor{currentfill}%
\pgfsetlinewidth{0.501875pt}%
\definecolor{currentstroke}{rgb}{0.000000,0.000000,0.000000}%
\pgfsetstrokecolor{currentstroke}%
\pgfsetdash{}{0pt}%
\pgfsys@defobject{currentmarker}{\pgfqpoint{0.000000in}{-0.055556in}}{\pgfqpoint{0.000000in}{0.000000in}}{%
\pgfpathmoveto{\pgfqpoint{0.000000in}{0.000000in}}%
\pgfpathlineto{\pgfqpoint{0.000000in}{-0.055556in}}%
\pgfusepath{stroke,fill}%
}%
\begin{pgfscope}%
\pgfsys@transformshift{4.381087in}{2.148139in}%
\pgfsys@useobject{currentmarker}{}%
\end{pgfscope}%
\end{pgfscope}%
\begin{pgfscope}%
\pgftext[x=4.381087in,y=0.434277in,,top]{{\rmfamily\fontsize{11.000000}{13.200000}\selectfont 2.1}}%
\end{pgfscope}%
\begin{pgfscope}%
\pgfsetbuttcap%
\pgfsetroundjoin%
\definecolor{currentfill}{rgb}{0.000000,0.000000,0.000000}%
\pgfsetfillcolor{currentfill}%
\pgfsetlinewidth{0.501875pt}%
\definecolor{currentstroke}{rgb}{0.000000,0.000000,0.000000}%
\pgfsetstrokecolor{currentstroke}%
\pgfsetdash{}{0pt}%
\pgfsys@defobject{currentmarker}{\pgfqpoint{0.000000in}{0.000000in}}{\pgfqpoint{0.000000in}{0.055556in}}{%
\pgfpathmoveto{\pgfqpoint{0.000000in}{0.000000in}}%
\pgfpathlineto{\pgfqpoint{0.000000in}{0.055556in}}%
\pgfusepath{stroke,fill}%
}%
\begin{pgfscope}%
\pgfsys@transformshift{4.996715in}{0.489833in}%
\pgfsys@useobject{currentmarker}{}%
\end{pgfscope}%
\end{pgfscope}%
\begin{pgfscope}%
\pgfsetbuttcap%
\pgfsetroundjoin%
\definecolor{currentfill}{rgb}{0.000000,0.000000,0.000000}%
\pgfsetfillcolor{currentfill}%
\pgfsetlinewidth{0.501875pt}%
\definecolor{currentstroke}{rgb}{0.000000,0.000000,0.000000}%
\pgfsetstrokecolor{currentstroke}%
\pgfsetdash{}{0pt}%
\pgfsys@defobject{currentmarker}{\pgfqpoint{0.000000in}{-0.055556in}}{\pgfqpoint{0.000000in}{0.000000in}}{%
\pgfpathmoveto{\pgfqpoint{0.000000in}{0.000000in}}%
\pgfpathlineto{\pgfqpoint{0.000000in}{-0.055556in}}%
\pgfusepath{stroke,fill}%
}%
\begin{pgfscope}%
\pgfsys@transformshift{4.996715in}{2.148139in}%
\pgfsys@useobject{currentmarker}{}%
\end{pgfscope}%
\end{pgfscope}%
\begin{pgfscope}%
\pgftext[x=4.996715in,y=0.434277in,,top]{{\rmfamily\fontsize{11.000000}{13.200000}\selectfont 2.2}}%
\end{pgfscope}%
\begin{pgfscope}%
\pgfsetbuttcap%
\pgfsetroundjoin%
\definecolor{currentfill}{rgb}{0.000000,0.000000,0.000000}%
\pgfsetfillcolor{currentfill}%
\pgfsetlinewidth{0.501875pt}%
\definecolor{currentstroke}{rgb}{0.000000,0.000000,0.000000}%
\pgfsetstrokecolor{currentstroke}%
\pgfsetdash{}{0pt}%
\pgfsys@defobject{currentmarker}{\pgfqpoint{0.000000in}{0.000000in}}{\pgfqpoint{0.000000in}{0.055556in}}{%
\pgfpathmoveto{\pgfqpoint{0.000000in}{0.000000in}}%
\pgfpathlineto{\pgfqpoint{0.000000in}{0.055556in}}%
\pgfusepath{stroke,fill}%
}%
\begin{pgfscope}%
\pgfsys@transformshift{5.612344in}{0.489833in}%
\pgfsys@useobject{currentmarker}{}%
\end{pgfscope}%
\end{pgfscope}%
\begin{pgfscope}%
\pgfsetbuttcap%
\pgfsetroundjoin%
\definecolor{currentfill}{rgb}{0.000000,0.000000,0.000000}%
\pgfsetfillcolor{currentfill}%
\pgfsetlinewidth{0.501875pt}%
\definecolor{currentstroke}{rgb}{0.000000,0.000000,0.000000}%
\pgfsetstrokecolor{currentstroke}%
\pgfsetdash{}{0pt}%
\pgfsys@defobject{currentmarker}{\pgfqpoint{0.000000in}{-0.055556in}}{\pgfqpoint{0.000000in}{0.000000in}}{%
\pgfpathmoveto{\pgfqpoint{0.000000in}{0.000000in}}%
\pgfpathlineto{\pgfqpoint{0.000000in}{-0.055556in}}%
\pgfusepath{stroke,fill}%
}%
\begin{pgfscope}%
\pgfsys@transformshift{5.612344in}{2.148139in}%
\pgfsys@useobject{currentmarker}{}%
\end{pgfscope}%
\end{pgfscope}%
\begin{pgfscope}%
\pgftext[x=5.612344in,y=0.434277in,,top]{{\rmfamily\fontsize{11.000000}{13.200000}\selectfont 2.3}}%
\end{pgfscope}%
\begin{pgfscope}%
\pgfsetbuttcap%
\pgfsetroundjoin%
\definecolor{currentfill}{rgb}{0.000000,0.000000,0.000000}%
\pgfsetfillcolor{currentfill}%
\pgfsetlinewidth{0.501875pt}%
\definecolor{currentstroke}{rgb}{0.000000,0.000000,0.000000}%
\pgfsetstrokecolor{currentstroke}%
\pgfsetdash{}{0pt}%
\pgfsys@defobject{currentmarker}{\pgfqpoint{0.000000in}{0.000000in}}{\pgfqpoint{0.000000in}{0.055556in}}{%
\pgfpathmoveto{\pgfqpoint{0.000000in}{0.000000in}}%
\pgfpathlineto{\pgfqpoint{0.000000in}{0.055556in}}%
\pgfusepath{stroke,fill}%
}%
\begin{pgfscope}%
\pgfsys@transformshift{6.227972in}{0.489833in}%
\pgfsys@useobject{currentmarker}{}%
\end{pgfscope}%
\end{pgfscope}%
\begin{pgfscope}%
\pgfsetbuttcap%
\pgfsetroundjoin%
\definecolor{currentfill}{rgb}{0.000000,0.000000,0.000000}%
\pgfsetfillcolor{currentfill}%
\pgfsetlinewidth{0.501875pt}%
\definecolor{currentstroke}{rgb}{0.000000,0.000000,0.000000}%
\pgfsetstrokecolor{currentstroke}%
\pgfsetdash{}{0pt}%
\pgfsys@defobject{currentmarker}{\pgfqpoint{0.000000in}{-0.055556in}}{\pgfqpoint{0.000000in}{0.000000in}}{%
\pgfpathmoveto{\pgfqpoint{0.000000in}{0.000000in}}%
\pgfpathlineto{\pgfqpoint{0.000000in}{-0.055556in}}%
\pgfusepath{stroke,fill}%
}%
\begin{pgfscope}%
\pgfsys@transformshift{6.227972in}{2.148139in}%
\pgfsys@useobject{currentmarker}{}%
\end{pgfscope}%
\end{pgfscope}%
\begin{pgfscope}%
\pgftext[x=6.227972in,y=0.434277in,,top]{{\rmfamily\fontsize{11.000000}{13.200000}\selectfont 2.4}}%
\end{pgfscope}%
\begin{pgfscope}%
\pgftext[x=4.996715in,y=0.229166in,,top]{{\rmfamily\fontsize{11.000000}{13.200000}\selectfont \(\displaystyle k_BT\)   [  \(\displaystyle J\) ]}}%
\end{pgfscope}%
\begin{pgfscope}%
\pgfsetbuttcap%
\pgfsetroundjoin%
\definecolor{currentfill}{rgb}{0.000000,0.000000,0.000000}%
\pgfsetfillcolor{currentfill}%
\pgfsetlinewidth{0.501875pt}%
\definecolor{currentstroke}{rgb}{0.000000,0.000000,0.000000}%
\pgfsetstrokecolor{currentstroke}%
\pgfsetdash{}{0pt}%
\pgfsys@defobject{currentmarker}{\pgfqpoint{0.000000in}{0.000000in}}{\pgfqpoint{0.055556in}{0.000000in}}{%
\pgfpathmoveto{\pgfqpoint{0.000000in}{0.000000in}}%
\pgfpathlineto{\pgfqpoint{0.055556in}{0.000000in}}%
\pgfusepath{stroke,fill}%
}%
\begin{pgfscope}%
\pgfsys@transformshift{3.765458in}{0.489833in}%
\pgfsys@useobject{currentmarker}{}%
\end{pgfscope}%
\end{pgfscope}%
\begin{pgfscope}%
\pgfsetbuttcap%
\pgfsetroundjoin%
\definecolor{currentfill}{rgb}{0.000000,0.000000,0.000000}%
\pgfsetfillcolor{currentfill}%
\pgfsetlinewidth{0.501875pt}%
\definecolor{currentstroke}{rgb}{0.000000,0.000000,0.000000}%
\pgfsetstrokecolor{currentstroke}%
\pgfsetdash{}{0pt}%
\pgfsys@defobject{currentmarker}{\pgfqpoint{-0.055556in}{0.000000in}}{\pgfqpoint{0.000000in}{0.000000in}}{%
\pgfpathmoveto{\pgfqpoint{0.000000in}{0.000000in}}%
\pgfpathlineto{\pgfqpoint{-0.055556in}{0.000000in}}%
\pgfusepath{stroke,fill}%
}%
\begin{pgfscope}%
\pgfsys@transformshift{6.227972in}{0.489833in}%
\pgfsys@useobject{currentmarker}{}%
\end{pgfscope}%
\end{pgfscope}%
\begin{pgfscope}%
\pgftext[x=3.709903in,y=0.489833in,right,]{{\rmfamily\fontsize{11.000000}{13.200000}\selectfont 0.0}}%
\end{pgfscope}%
\begin{pgfscope}%
\pgfsetbuttcap%
\pgfsetroundjoin%
\definecolor{currentfill}{rgb}{0.000000,0.000000,0.000000}%
\pgfsetfillcolor{currentfill}%
\pgfsetlinewidth{0.501875pt}%
\definecolor{currentstroke}{rgb}{0.000000,0.000000,0.000000}%
\pgfsetstrokecolor{currentstroke}%
\pgfsetdash{}{0pt}%
\pgfsys@defobject{currentmarker}{\pgfqpoint{0.000000in}{0.000000in}}{\pgfqpoint{0.055556in}{0.000000in}}{%
\pgfpathmoveto{\pgfqpoint{0.000000in}{0.000000in}}%
\pgfpathlineto{\pgfqpoint{0.055556in}{0.000000in}}%
\pgfusepath{stroke,fill}%
}%
\begin{pgfscope}%
\pgfsys@transformshift{3.765458in}{0.821494in}%
\pgfsys@useobject{currentmarker}{}%
\end{pgfscope}%
\end{pgfscope}%
\begin{pgfscope}%
\pgfsetbuttcap%
\pgfsetroundjoin%
\definecolor{currentfill}{rgb}{0.000000,0.000000,0.000000}%
\pgfsetfillcolor{currentfill}%
\pgfsetlinewidth{0.501875pt}%
\definecolor{currentstroke}{rgb}{0.000000,0.000000,0.000000}%
\pgfsetstrokecolor{currentstroke}%
\pgfsetdash{}{0pt}%
\pgfsys@defobject{currentmarker}{\pgfqpoint{-0.055556in}{0.000000in}}{\pgfqpoint{0.000000in}{0.000000in}}{%
\pgfpathmoveto{\pgfqpoint{0.000000in}{0.000000in}}%
\pgfpathlineto{\pgfqpoint{-0.055556in}{0.000000in}}%
\pgfusepath{stroke,fill}%
}%
\begin{pgfscope}%
\pgfsys@transformshift{6.227972in}{0.821494in}%
\pgfsys@useobject{currentmarker}{}%
\end{pgfscope}%
\end{pgfscope}%
\begin{pgfscope}%
\pgftext[x=3.709903in,y=0.821494in,right,]{{\rmfamily\fontsize{11.000000}{13.200000}\selectfont 0.2}}%
\end{pgfscope}%
\begin{pgfscope}%
\pgfsetbuttcap%
\pgfsetroundjoin%
\definecolor{currentfill}{rgb}{0.000000,0.000000,0.000000}%
\pgfsetfillcolor{currentfill}%
\pgfsetlinewidth{0.501875pt}%
\definecolor{currentstroke}{rgb}{0.000000,0.000000,0.000000}%
\pgfsetstrokecolor{currentstroke}%
\pgfsetdash{}{0pt}%
\pgfsys@defobject{currentmarker}{\pgfqpoint{0.000000in}{0.000000in}}{\pgfqpoint{0.055556in}{0.000000in}}{%
\pgfpathmoveto{\pgfqpoint{0.000000in}{0.000000in}}%
\pgfpathlineto{\pgfqpoint{0.055556in}{0.000000in}}%
\pgfusepath{stroke,fill}%
}%
\begin{pgfscope}%
\pgfsys@transformshift{3.765458in}{1.153155in}%
\pgfsys@useobject{currentmarker}{}%
\end{pgfscope}%
\end{pgfscope}%
\begin{pgfscope}%
\pgfsetbuttcap%
\pgfsetroundjoin%
\definecolor{currentfill}{rgb}{0.000000,0.000000,0.000000}%
\pgfsetfillcolor{currentfill}%
\pgfsetlinewidth{0.501875pt}%
\definecolor{currentstroke}{rgb}{0.000000,0.000000,0.000000}%
\pgfsetstrokecolor{currentstroke}%
\pgfsetdash{}{0pt}%
\pgfsys@defobject{currentmarker}{\pgfqpoint{-0.055556in}{0.000000in}}{\pgfqpoint{0.000000in}{0.000000in}}{%
\pgfpathmoveto{\pgfqpoint{0.000000in}{0.000000in}}%
\pgfpathlineto{\pgfqpoint{-0.055556in}{0.000000in}}%
\pgfusepath{stroke,fill}%
}%
\begin{pgfscope}%
\pgfsys@transformshift{6.227972in}{1.153155in}%
\pgfsys@useobject{currentmarker}{}%
\end{pgfscope}%
\end{pgfscope}%
\begin{pgfscope}%
\pgftext[x=3.709903in,y=1.153155in,right,]{{\rmfamily\fontsize{11.000000}{13.200000}\selectfont 0.4}}%
\end{pgfscope}%
\begin{pgfscope}%
\pgfsetbuttcap%
\pgfsetroundjoin%
\definecolor{currentfill}{rgb}{0.000000,0.000000,0.000000}%
\pgfsetfillcolor{currentfill}%
\pgfsetlinewidth{0.501875pt}%
\definecolor{currentstroke}{rgb}{0.000000,0.000000,0.000000}%
\pgfsetstrokecolor{currentstroke}%
\pgfsetdash{}{0pt}%
\pgfsys@defobject{currentmarker}{\pgfqpoint{0.000000in}{0.000000in}}{\pgfqpoint{0.055556in}{0.000000in}}{%
\pgfpathmoveto{\pgfqpoint{0.000000in}{0.000000in}}%
\pgfpathlineto{\pgfqpoint{0.055556in}{0.000000in}}%
\pgfusepath{stroke,fill}%
}%
\begin{pgfscope}%
\pgfsys@transformshift{3.765458in}{1.484816in}%
\pgfsys@useobject{currentmarker}{}%
\end{pgfscope}%
\end{pgfscope}%
\begin{pgfscope}%
\pgfsetbuttcap%
\pgfsetroundjoin%
\definecolor{currentfill}{rgb}{0.000000,0.000000,0.000000}%
\pgfsetfillcolor{currentfill}%
\pgfsetlinewidth{0.501875pt}%
\definecolor{currentstroke}{rgb}{0.000000,0.000000,0.000000}%
\pgfsetstrokecolor{currentstroke}%
\pgfsetdash{}{0pt}%
\pgfsys@defobject{currentmarker}{\pgfqpoint{-0.055556in}{0.000000in}}{\pgfqpoint{0.000000in}{0.000000in}}{%
\pgfpathmoveto{\pgfqpoint{0.000000in}{0.000000in}}%
\pgfpathlineto{\pgfqpoint{-0.055556in}{0.000000in}}%
\pgfusepath{stroke,fill}%
}%
\begin{pgfscope}%
\pgfsys@transformshift{6.227972in}{1.484816in}%
\pgfsys@useobject{currentmarker}{}%
\end{pgfscope}%
\end{pgfscope}%
\begin{pgfscope}%
\pgftext[x=3.709903in,y=1.484816in,right,]{{\rmfamily\fontsize{11.000000}{13.200000}\selectfont 0.6}}%
\end{pgfscope}%
\begin{pgfscope}%
\pgfsetbuttcap%
\pgfsetroundjoin%
\definecolor{currentfill}{rgb}{0.000000,0.000000,0.000000}%
\pgfsetfillcolor{currentfill}%
\pgfsetlinewidth{0.501875pt}%
\definecolor{currentstroke}{rgb}{0.000000,0.000000,0.000000}%
\pgfsetstrokecolor{currentstroke}%
\pgfsetdash{}{0pt}%
\pgfsys@defobject{currentmarker}{\pgfqpoint{0.000000in}{0.000000in}}{\pgfqpoint{0.055556in}{0.000000in}}{%
\pgfpathmoveto{\pgfqpoint{0.000000in}{0.000000in}}%
\pgfpathlineto{\pgfqpoint{0.055556in}{0.000000in}}%
\pgfusepath{stroke,fill}%
}%
\begin{pgfscope}%
\pgfsys@transformshift{3.765458in}{1.816478in}%
\pgfsys@useobject{currentmarker}{}%
\end{pgfscope}%
\end{pgfscope}%
\begin{pgfscope}%
\pgfsetbuttcap%
\pgfsetroundjoin%
\definecolor{currentfill}{rgb}{0.000000,0.000000,0.000000}%
\pgfsetfillcolor{currentfill}%
\pgfsetlinewidth{0.501875pt}%
\definecolor{currentstroke}{rgb}{0.000000,0.000000,0.000000}%
\pgfsetstrokecolor{currentstroke}%
\pgfsetdash{}{0pt}%
\pgfsys@defobject{currentmarker}{\pgfqpoint{-0.055556in}{0.000000in}}{\pgfqpoint{0.000000in}{0.000000in}}{%
\pgfpathmoveto{\pgfqpoint{0.000000in}{0.000000in}}%
\pgfpathlineto{\pgfqpoint{-0.055556in}{0.000000in}}%
\pgfusepath{stroke,fill}%
}%
\begin{pgfscope}%
\pgfsys@transformshift{6.227972in}{1.816478in}%
\pgfsys@useobject{currentmarker}{}%
\end{pgfscope}%
\end{pgfscope}%
\begin{pgfscope}%
\pgftext[x=3.709903in,y=1.816478in,right,]{{\rmfamily\fontsize{11.000000}{13.200000}\selectfont 0.8}}%
\end{pgfscope}%
\begin{pgfscope}%
\pgfsetbuttcap%
\pgfsetroundjoin%
\definecolor{currentfill}{rgb}{0.000000,0.000000,0.000000}%
\pgfsetfillcolor{currentfill}%
\pgfsetlinewidth{0.501875pt}%
\definecolor{currentstroke}{rgb}{0.000000,0.000000,0.000000}%
\pgfsetstrokecolor{currentstroke}%
\pgfsetdash{}{0pt}%
\pgfsys@defobject{currentmarker}{\pgfqpoint{0.000000in}{0.000000in}}{\pgfqpoint{0.055556in}{0.000000in}}{%
\pgfpathmoveto{\pgfqpoint{0.000000in}{0.000000in}}%
\pgfpathlineto{\pgfqpoint{0.055556in}{0.000000in}}%
\pgfusepath{stroke,fill}%
}%
\begin{pgfscope}%
\pgfsys@transformshift{3.765458in}{2.148139in}%
\pgfsys@useobject{currentmarker}{}%
\end{pgfscope}%
\end{pgfscope}%
\begin{pgfscope}%
\pgfsetbuttcap%
\pgfsetroundjoin%
\definecolor{currentfill}{rgb}{0.000000,0.000000,0.000000}%
\pgfsetfillcolor{currentfill}%
\pgfsetlinewidth{0.501875pt}%
\definecolor{currentstroke}{rgb}{0.000000,0.000000,0.000000}%
\pgfsetstrokecolor{currentstroke}%
\pgfsetdash{}{0pt}%
\pgfsys@defobject{currentmarker}{\pgfqpoint{-0.055556in}{0.000000in}}{\pgfqpoint{0.000000in}{0.000000in}}{%
\pgfpathmoveto{\pgfqpoint{0.000000in}{0.000000in}}%
\pgfpathlineto{\pgfqpoint{-0.055556in}{0.000000in}}%
\pgfusepath{stroke,fill}%
}%
\begin{pgfscope}%
\pgfsys@transformshift{6.227972in}{2.148139in}%
\pgfsys@useobject{currentmarker}{}%
\end{pgfscope}%
\end{pgfscope}%
\begin{pgfscope}%
\pgftext[x=3.709903in,y=2.148139in,right,]{{\rmfamily\fontsize{11.000000}{13.200000}\selectfont 1.0}}%
\end{pgfscope}%
\begin{pgfscope}%
\pgftext[x=3.449180in,y=1.318986in,,bottom,rotate=90.000000]{{\rmfamily\fontsize{11.000000}{13.200000}\selectfont \(\displaystyle \langle |M| \rangle\)   [ \(\displaystyle \cdot\) ]}}%
\end{pgfscope}%
\begin{pgfscope}%
\pgftext[x=4.996715in,y=2.217583in,,base]{{\rmfamily\fontsize{11.000000}{13.200000}\selectfont Magnetization vs thermal energy}}%
\end{pgfscope}%
\end{pgfpicture}%
\makeatother%
\endgroup%
\\
%% Creator: Matplotlib, PGF backend
%%
%% To include the figure in your LaTeX document, write
%%   \input{<filename>.pgf}
%%
%% Make sure the required packages are loaded in your preamble
%%   \usepackage{pgf}
%%
%% Figures using additional raster images can only be included by \input if
%% they are in the same directory as the main LaTeX file. For loading figures
%% from other directories you can use the `import` package
%%   \usepackage{import}
%% and then include the figures with
%%   \import{<path to file>}{<filename>.pgf}
%%
%% Matplotlib used the following preamble
%%   \usepackage[utf8]{inputenc}
%%   \usepackage[T1]{fontenc}
%%   \usepackage{cmbright}
%%   \usepackage{newtxtext}
%%   \usepackage{bm}
%%   \usepackage{amsmath,amsthm}
%%   \usepackage{fontspec}
%%   \setsansfont{DejaVu Sans}
%%   \setmonofont{DejaVu Sans Mono}
%%
\begingroup%
\makeatletter%
\begin{pgfpicture}%
\pgfpathrectangle{\pgfpointorigin}{\pgfqpoint{6.400000in}{2.400000in}}%
\pgfusepath{use as bounding box, clip}%
\begin{pgfscope}%
\pgfsetbuttcap%
\pgfsetmiterjoin%
\definecolor{currentfill}{rgb}{1.000000,1.000000,1.000000}%
\pgfsetfillcolor{currentfill}%
\pgfsetlinewidth{0.000000pt}%
\definecolor{currentstroke}{rgb}{1.000000,1.000000,1.000000}%
\pgfsetstrokecolor{currentstroke}%
\pgfsetdash{}{0pt}%
\pgfpathmoveto{\pgfqpoint{0.000000in}{0.000000in}}%
\pgfpathlineto{\pgfqpoint{6.400000in}{0.000000in}}%
\pgfpathlineto{\pgfqpoint{6.400000in}{2.400000in}}%
\pgfpathlineto{\pgfqpoint{0.000000in}{2.400000in}}%
\pgfpathclose%
\pgfusepath{fill}%
\end{pgfscope}%
\begin{pgfscope}%
\pgfsetbuttcap%
\pgfsetmiterjoin%
\definecolor{currentfill}{rgb}{1.000000,1.000000,1.000000}%
\pgfsetfillcolor{currentfill}%
\pgfsetlinewidth{0.000000pt}%
\definecolor{currentstroke}{rgb}{0.000000,0.000000,0.000000}%
\pgfsetstrokecolor{currentstroke}%
\pgfsetstrokeopacity{0.000000}%
\pgfsetdash{}{0pt}%
\pgfpathmoveto{\pgfqpoint{0.545444in}{0.489833in}}%
\pgfpathlineto{\pgfqpoint{2.967626in}{0.489833in}}%
\pgfpathlineto{\pgfqpoint{2.967626in}{2.148139in}}%
\pgfpathlineto{\pgfqpoint{0.545444in}{2.148139in}}%
\pgfpathclose%
\pgfusepath{fill}%
\end{pgfscope}%
\begin{pgfscope}%
\pgfpathrectangle{\pgfqpoint{0.545444in}{0.489833in}}{\pgfqpoint{2.422182in}{1.658306in}} %
\pgfusepath{clip}%
\pgfsetbuttcap%
\pgfsetroundjoin%
\pgfsetlinewidth{1.003750pt}%
\definecolor{currentstroke}{rgb}{0.000000,0.000000,0.000000}%
\pgfsetstrokecolor{currentstroke}%
\pgfsetdash{{8.000000pt}{4.000000pt}{2.000000pt}{4.000000pt}{2.000000pt}{4.000000pt}}{0.000000pt}%
\pgfpathmoveto{\pgfqpoint{0.545444in}{0.638203in}}%
\pgfpathlineto{\pgfqpoint{0.787662in}{0.695609in}}%
\pgfpathlineto{\pgfqpoint{0.920882in}{0.729647in}}%
\pgfpathlineto{\pgfqpoint{1.029881in}{0.759820in}}%
\pgfpathlineto{\pgfqpoint{1.126768in}{0.788939in}}%
\pgfpathlineto{\pgfqpoint{1.211544in}{0.816552in}}%
\pgfpathlineto{\pgfqpoint{1.296321in}{0.846465in}}%
\pgfpathlineto{\pgfqpoint{1.381097in}{0.878868in}}%
\pgfpathlineto{\pgfqpoint{1.465873in}{0.913593in}}%
\pgfpathlineto{\pgfqpoint{1.562760in}{0.955815in}}%
\pgfpathlineto{\pgfqpoint{1.659648in}{1.000400in}}%
\pgfpathlineto{\pgfqpoint{1.804979in}{1.070197in}}%
\pgfpathlineto{\pgfqpoint{1.950310in}{1.139613in}}%
\pgfpathlineto{\pgfqpoint{2.022975in}{1.171653in}}%
\pgfpathlineto{\pgfqpoint{2.083530in}{1.195623in}}%
\pgfpathlineto{\pgfqpoint{2.131973in}{1.212681in}}%
\pgfpathlineto{\pgfqpoint{2.180417in}{1.227788in}}%
\pgfpathlineto{\pgfqpoint{2.228860in}{1.240820in}}%
\pgfpathlineto{\pgfqpoint{2.277304in}{1.251406in}}%
\pgfpathlineto{\pgfqpoint{2.313637in}{1.257365in}}%
\pgfpathlineto{\pgfqpoint{2.349970in}{1.261420in}}%
\pgfpathlineto{\pgfqpoint{2.386302in}{1.263592in}}%
\pgfpathlineto{\pgfqpoint{2.422635in}{1.263933in}}%
\pgfpathlineto{\pgfqpoint{2.458968in}{1.262427in}}%
\pgfpathlineto{\pgfqpoint{2.495300in}{1.258974in}}%
\pgfpathlineto{\pgfqpoint{2.531633in}{1.253652in}}%
\pgfpathlineto{\pgfqpoint{2.580077in}{1.243952in}}%
\pgfpathlineto{\pgfqpoint{2.628520in}{1.231643in}}%
\pgfpathlineto{\pgfqpoint{2.676964in}{1.216975in}}%
\pgfpathlineto{\pgfqpoint{2.725408in}{1.200120in}}%
\pgfpathlineto{\pgfqpoint{2.785962in}{1.176312in}}%
\pgfpathlineto{\pgfqpoint{2.858628in}{1.144642in}}%
\pgfpathlineto{\pgfqpoint{2.967626in}{1.094388in}}%
\pgfpathlineto{\pgfqpoint{2.967626in}{1.094388in}}%
\pgfusepath{stroke}%
\end{pgfscope}%
\begin{pgfscope}%
\pgfpathrectangle{\pgfqpoint{0.545444in}{0.489833in}}{\pgfqpoint{2.422182in}{1.658306in}} %
\pgfusepath{clip}%
\pgfsetbuttcap%
\pgfsetroundjoin%
\pgfsetlinewidth{1.003750pt}%
\definecolor{currentstroke}{rgb}{0.000000,0.000000,0.000000}%
\pgfsetstrokecolor{currentstroke}%
\pgfsetdash{{6.000000pt}{6.000000pt}}{0.000000pt}%
\pgfpathmoveto{\pgfqpoint{0.545444in}{0.638161in}}%
\pgfpathlineto{\pgfqpoint{0.763441in}{0.688307in}}%
\pgfpathlineto{\pgfqpoint{0.884550in}{0.718494in}}%
\pgfpathlineto{\pgfqpoint{0.981437in}{0.744768in}}%
\pgfpathlineto{\pgfqpoint{1.078324in}{0.773416in}}%
\pgfpathlineto{\pgfqpoint{1.163101in}{0.800720in}}%
\pgfpathlineto{\pgfqpoint{1.247877in}{0.830374in}}%
\pgfpathlineto{\pgfqpoint{1.332653in}{0.862639in}}%
\pgfpathlineto{\pgfqpoint{1.405319in}{0.892840in}}%
\pgfpathlineto{\pgfqpoint{1.465873in}{0.920290in}}%
\pgfpathlineto{\pgfqpoint{1.526428in}{0.950220in}}%
\pgfpathlineto{\pgfqpoint{1.586982in}{0.983007in}}%
\pgfpathlineto{\pgfqpoint{1.635426in}{1.011547in}}%
\pgfpathlineto{\pgfqpoint{1.683870in}{1.042454in}}%
\pgfpathlineto{\pgfqpoint{1.720202in}{1.067725in}}%
\pgfpathlineto{\pgfqpoint{1.756535in}{1.095324in}}%
\pgfpathlineto{\pgfqpoint{1.792868in}{1.125735in}}%
\pgfpathlineto{\pgfqpoint{1.829200in}{1.159441in}}%
\pgfpathlineto{\pgfqpoint{1.865533in}{1.196926in}}%
\pgfpathlineto{\pgfqpoint{1.901866in}{1.238566in}}%
\pgfpathlineto{\pgfqpoint{1.938199in}{1.283756in}}%
\pgfpathlineto{\pgfqpoint{1.986642in}{1.347572in}}%
\pgfpathlineto{\pgfqpoint{2.083530in}{1.479887in}}%
\pgfpathlineto{\pgfqpoint{2.131973in}{1.545461in}}%
\pgfpathlineto{\pgfqpoint{2.156195in}{1.573603in}}%
\pgfpathlineto{\pgfqpoint{2.180417in}{1.596556in}}%
\pgfpathlineto{\pgfqpoint{2.192528in}{1.605874in}}%
\pgfpathlineto{\pgfqpoint{2.204639in}{1.613651in}}%
\pgfpathlineto{\pgfqpoint{2.216750in}{1.619861in}}%
\pgfpathlineto{\pgfqpoint{2.228860in}{1.624485in}}%
\pgfpathlineto{\pgfqpoint{2.240971in}{1.627509in}}%
\pgfpathlineto{\pgfqpoint{2.253082in}{1.628912in}}%
\pgfpathlineto{\pgfqpoint{2.265193in}{1.628647in}}%
\pgfpathlineto{\pgfqpoint{2.277304in}{1.626640in}}%
\pgfpathlineto{\pgfqpoint{2.289415in}{1.622820in}}%
\pgfpathlineto{\pgfqpoint{2.301526in}{1.617114in}}%
\pgfpathlineto{\pgfqpoint{2.313637in}{1.609502in}}%
\pgfpathlineto{\pgfqpoint{2.325748in}{1.600113in}}%
\pgfpathlineto{\pgfqpoint{2.349970in}{1.576617in}}%
\pgfpathlineto{\pgfqpoint{2.374191in}{1.547811in}}%
\pgfpathlineto{\pgfqpoint{2.398413in}{1.514719in}}%
\pgfpathlineto{\pgfqpoint{2.434746in}{1.459234in}}%
\pgfpathlineto{\pgfqpoint{2.543744in}{1.284899in}}%
\pgfpathlineto{\pgfqpoint{2.580077in}{1.231997in}}%
\pgfpathlineto{\pgfqpoint{2.604299in}{1.199652in}}%
\pgfpathlineto{\pgfqpoint{2.640631in}{1.155919in}}%
\pgfpathlineto{\pgfqpoint{2.676964in}{1.116482in}}%
\pgfpathlineto{\pgfqpoint{2.713297in}{1.080386in}}%
\pgfpathlineto{\pgfqpoint{2.749630in}{1.047727in}}%
\pgfpathlineto{\pgfqpoint{2.785962in}{1.018662in}}%
\pgfpathlineto{\pgfqpoint{2.822295in}{0.993034in}}%
\pgfpathlineto{\pgfqpoint{2.858628in}{0.970226in}}%
\pgfpathlineto{\pgfqpoint{2.907071in}{0.943045in}}%
\pgfpathlineto{\pgfqpoint{2.967626in}{0.911732in}}%
\pgfpathlineto{\pgfqpoint{2.967626in}{0.911732in}}%
\pgfusepath{stroke}%
\end{pgfscope}%
\begin{pgfscope}%
\pgfpathrectangle{\pgfqpoint{0.545444in}{0.489833in}}{\pgfqpoint{2.422182in}{1.658306in}} %
\pgfusepath{clip}%
\pgfsetbuttcap%
\pgfsetroundjoin%
\pgfsetlinewidth{1.003750pt}%
\definecolor{currentstroke}{rgb}{0.000000,0.000000,0.000000}%
\pgfsetstrokecolor{currentstroke}%
\pgfsetdash{{2.000000pt}{2.000000pt}}{0.000000pt}%
\pgfpathmoveto{\pgfqpoint{0.545444in}{0.638023in}}%
\pgfpathlineto{\pgfqpoint{0.763441in}{0.688347in}}%
\pgfpathlineto{\pgfqpoint{0.884550in}{0.718501in}}%
\pgfpathlineto{\pgfqpoint{0.981437in}{0.744638in}}%
\pgfpathlineto{\pgfqpoint{1.078324in}{0.773125in}}%
\pgfpathlineto{\pgfqpoint{1.163101in}{0.800384in}}%
\pgfpathlineto{\pgfqpoint{1.247877in}{0.830180in}}%
\pgfpathlineto{\pgfqpoint{1.320542in}{0.858002in}}%
\pgfpathlineto{\pgfqpoint{1.393208in}{0.888287in}}%
\pgfpathlineto{\pgfqpoint{1.453762in}{0.915766in}}%
\pgfpathlineto{\pgfqpoint{1.514317in}{0.945601in}}%
\pgfpathlineto{\pgfqpoint{1.574871in}{0.978086in}}%
\pgfpathlineto{\pgfqpoint{1.635426in}{1.013518in}}%
\pgfpathlineto{\pgfqpoint{1.683870in}{1.044297in}}%
\pgfpathlineto{\pgfqpoint{1.720202in}{1.069369in}}%
\pgfpathlineto{\pgfqpoint{1.756535in}{1.096703in}}%
\pgfpathlineto{\pgfqpoint{1.792868in}{1.126803in}}%
\pgfpathlineto{\pgfqpoint{1.829200in}{1.160172in}}%
\pgfpathlineto{\pgfqpoint{1.865533in}{1.197313in}}%
\pgfpathlineto{\pgfqpoint{1.901866in}{1.238769in}}%
\pgfpathlineto{\pgfqpoint{1.926088in}{1.269244in}}%
\pgfpathlineto{\pgfqpoint{1.950310in}{1.302372in}}%
\pgfpathlineto{\pgfqpoint{1.974531in}{1.338486in}}%
\pgfpathlineto{\pgfqpoint{1.998753in}{1.377917in}}%
\pgfpathlineto{\pgfqpoint{2.022975in}{1.420997in}}%
\pgfpathlineto{\pgfqpoint{2.047197in}{1.468071in}}%
\pgfpathlineto{\pgfqpoint{2.071419in}{1.519486in}}%
\pgfpathlineto{\pgfqpoint{2.095640in}{1.575555in}}%
\pgfpathlineto{\pgfqpoint{2.144084in}{1.692597in}}%
\pgfpathlineto{\pgfqpoint{2.168306in}{1.744057in}}%
\pgfpathlineto{\pgfqpoint{2.180417in}{1.765264in}}%
\pgfpathlineto{\pgfqpoint{2.192528in}{1.782268in}}%
\pgfpathlineto{\pgfqpoint{2.204639in}{1.794274in}}%
\pgfpathlineto{\pgfqpoint{2.216750in}{1.800905in}}%
\pgfpathlineto{\pgfqpoint{2.228860in}{1.801892in}}%
\pgfpathlineto{\pgfqpoint{2.240971in}{1.796967in}}%
\pgfpathlineto{\pgfqpoint{2.253082in}{1.785908in}}%
\pgfpathlineto{\pgfqpoint{2.265193in}{1.769192in}}%
\pgfpathlineto{\pgfqpoint{2.277304in}{1.747833in}}%
\pgfpathlineto{\pgfqpoint{2.289415in}{1.722861in}}%
\pgfpathlineto{\pgfqpoint{2.313637in}{1.666020in}}%
\pgfpathlineto{\pgfqpoint{2.349970in}{1.571300in}}%
\pgfpathlineto{\pgfqpoint{2.398413in}{1.441727in}}%
\pgfpathlineto{\pgfqpoint{2.422635in}{1.383052in}}%
\pgfpathlineto{\pgfqpoint{2.446857in}{1.331021in}}%
\pgfpathlineto{\pgfqpoint{2.471079in}{1.285250in}}%
\pgfpathlineto{\pgfqpoint{2.495300in}{1.245235in}}%
\pgfpathlineto{\pgfqpoint{2.519522in}{1.210305in}}%
\pgfpathlineto{\pgfqpoint{2.543744in}{1.179753in}}%
\pgfpathlineto{\pgfqpoint{2.567966in}{1.152712in}}%
\pgfpathlineto{\pgfqpoint{2.604299in}{1.116293in}}%
\pgfpathlineto{\pgfqpoint{2.640631in}{1.082828in}}%
\pgfpathlineto{\pgfqpoint{2.664853in}{1.062671in}}%
\pgfpathlineto{\pgfqpoint{2.689075in}{1.044606in}}%
\pgfpathlineto{\pgfqpoint{2.725408in}{1.020787in}}%
\pgfpathlineto{\pgfqpoint{2.761740in}{0.999753in}}%
\pgfpathlineto{\pgfqpoint{2.810184in}{0.974062in}}%
\pgfpathlineto{\pgfqpoint{2.858628in}{0.950566in}}%
\pgfpathlineto{\pgfqpoint{2.907071in}{0.929421in}}%
\pgfpathlineto{\pgfqpoint{2.967626in}{0.904887in}}%
\pgfpathlineto{\pgfqpoint{2.967626in}{0.904887in}}%
\pgfusepath{stroke}%
\end{pgfscope}%
\begin{pgfscope}%
\pgfpathrectangle{\pgfqpoint{0.545444in}{0.489833in}}{\pgfqpoint{2.422182in}{1.658306in}} %
\pgfusepath{clip}%
\pgfsetrectcap%
\pgfsetroundjoin%
\pgfsetlinewidth{1.003750pt}%
\definecolor{currentstroke}{rgb}{0.000000,0.000000,0.000000}%
\pgfsetstrokecolor{currentstroke}%
\pgfsetdash{}{0pt}%
\pgfpathmoveto{\pgfqpoint{0.545444in}{0.638766in}}%
\pgfpathlineto{\pgfqpoint{0.751330in}{0.684794in}}%
\pgfpathlineto{\pgfqpoint{0.860328in}{0.711395in}}%
\pgfpathlineto{\pgfqpoint{0.957215in}{0.737357in}}%
\pgfpathlineto{\pgfqpoint{1.041991in}{0.762216in}}%
\pgfpathlineto{\pgfqpoint{1.126768in}{0.789134in}}%
\pgfpathlineto{\pgfqpoint{1.211544in}{0.818150in}}%
\pgfpathlineto{\pgfqpoint{1.296321in}{0.849304in}}%
\pgfpathlineto{\pgfqpoint{1.368986in}{0.877916in}}%
\pgfpathlineto{\pgfqpoint{1.429541in}{0.903668in}}%
\pgfpathlineto{\pgfqpoint{1.490095in}{0.931705in}}%
\pgfpathlineto{\pgfqpoint{1.550650in}{0.962543in}}%
\pgfpathlineto{\pgfqpoint{1.599093in}{0.989575in}}%
\pgfpathlineto{\pgfqpoint{1.647537in}{1.018993in}}%
\pgfpathlineto{\pgfqpoint{1.695980in}{1.051191in}}%
\pgfpathlineto{\pgfqpoint{1.732313in}{1.077624in}}%
\pgfpathlineto{\pgfqpoint{1.768646in}{1.106405in}}%
\pgfpathlineto{\pgfqpoint{1.804979in}{1.137888in}}%
\pgfpathlineto{\pgfqpoint{1.841311in}{1.172424in}}%
\pgfpathlineto{\pgfqpoint{1.877644in}{1.210367in}}%
\pgfpathlineto{\pgfqpoint{1.901866in}{1.237908in}}%
\pgfpathlineto{\pgfqpoint{1.926088in}{1.268114in}}%
\pgfpathlineto{\pgfqpoint{1.950310in}{1.301902in}}%
\pgfpathlineto{\pgfqpoint{1.974531in}{1.340186in}}%
\pgfpathlineto{\pgfqpoint{1.998753in}{1.383883in}}%
\pgfpathlineto{\pgfqpoint{2.022975in}{1.433700in}}%
\pgfpathlineto{\pgfqpoint{2.047197in}{1.489493in}}%
\pgfpathlineto{\pgfqpoint{2.071419in}{1.550903in}}%
\pgfpathlineto{\pgfqpoint{2.095640in}{1.617561in}}%
\pgfpathlineto{\pgfqpoint{2.131973in}{1.725833in}}%
\pgfpathlineto{\pgfqpoint{2.168306in}{1.836355in}}%
\pgfpathlineto{\pgfqpoint{2.180417in}{1.866371in}}%
\pgfpathlineto{\pgfqpoint{2.192528in}{1.888673in}}%
\pgfpathlineto{\pgfqpoint{2.204639in}{1.901099in}}%
\pgfpathlineto{\pgfqpoint{2.216750in}{1.903362in}}%
\pgfpathlineto{\pgfqpoint{2.228860in}{1.895675in}}%
\pgfpathlineto{\pgfqpoint{2.240971in}{1.878248in}}%
\pgfpathlineto{\pgfqpoint{2.253082in}{1.851359in}}%
\pgfpathlineto{\pgfqpoint{2.265193in}{1.816271in}}%
\pgfpathlineto{\pgfqpoint{2.277304in}{1.775007in}}%
\pgfpathlineto{\pgfqpoint{2.301526in}{1.682129in}}%
\pgfpathlineto{\pgfqpoint{2.337859in}{1.542138in}}%
\pgfpathlineto{\pgfqpoint{2.362080in}{1.460751in}}%
\pgfpathlineto{\pgfqpoint{2.386302in}{1.393355in}}%
\pgfpathlineto{\pgfqpoint{2.410524in}{1.338122in}}%
\pgfpathlineto{\pgfqpoint{2.434746in}{1.292869in}}%
\pgfpathlineto{\pgfqpoint{2.458968in}{1.255090in}}%
\pgfpathlineto{\pgfqpoint{2.483190in}{1.222280in}}%
\pgfpathlineto{\pgfqpoint{2.507411in}{1.192981in}}%
\pgfpathlineto{\pgfqpoint{2.531633in}{1.166551in}}%
\pgfpathlineto{\pgfqpoint{2.567966in}{1.131094in}}%
\pgfpathlineto{\pgfqpoint{2.604299in}{1.099864in}}%
\pgfpathlineto{\pgfqpoint{2.640631in}{1.072251in}}%
\pgfpathlineto{\pgfqpoint{2.676964in}{1.047807in}}%
\pgfpathlineto{\pgfqpoint{2.713297in}{1.025900in}}%
\pgfpathlineto{\pgfqpoint{2.773851in}{0.992577in}}%
\pgfpathlineto{\pgfqpoint{2.846517in}{0.954844in}}%
\pgfpathlineto{\pgfqpoint{2.894960in}{0.932646in}}%
\pgfpathlineto{\pgfqpoint{2.955515in}{0.908000in}}%
\pgfpathlineto{\pgfqpoint{2.967626in}{0.903216in}}%
\pgfpathlineto{\pgfqpoint{2.967626in}{0.903216in}}%
\pgfusepath{stroke}%
\end{pgfscope}%
\begin{pgfscope}%
\pgfsetrectcap%
\pgfsetmiterjoin%
\pgfsetlinewidth{1.003750pt}%
\definecolor{currentstroke}{rgb}{0.000000,0.000000,0.000000}%
\pgfsetstrokecolor{currentstroke}%
\pgfsetdash{}{0pt}%
\pgfpathmoveto{\pgfqpoint{0.545444in}{2.148139in}}%
\pgfpathlineto{\pgfqpoint{2.967626in}{2.148139in}}%
\pgfusepath{stroke}%
\end{pgfscope}%
\begin{pgfscope}%
\pgfsetrectcap%
\pgfsetmiterjoin%
\pgfsetlinewidth{1.003750pt}%
\definecolor{currentstroke}{rgb}{0.000000,0.000000,0.000000}%
\pgfsetstrokecolor{currentstroke}%
\pgfsetdash{}{0pt}%
\pgfpathmoveto{\pgfqpoint{2.967626in}{0.489833in}}%
\pgfpathlineto{\pgfqpoint{2.967626in}{2.148139in}}%
\pgfusepath{stroke}%
\end{pgfscope}%
\begin{pgfscope}%
\pgfsetrectcap%
\pgfsetmiterjoin%
\pgfsetlinewidth{1.003750pt}%
\definecolor{currentstroke}{rgb}{0.000000,0.000000,0.000000}%
\pgfsetstrokecolor{currentstroke}%
\pgfsetdash{}{0pt}%
\pgfpathmoveto{\pgfqpoint{0.545444in}{0.489833in}}%
\pgfpathlineto{\pgfqpoint{2.967626in}{0.489833in}}%
\pgfusepath{stroke}%
\end{pgfscope}%
\begin{pgfscope}%
\pgfsetrectcap%
\pgfsetmiterjoin%
\pgfsetlinewidth{1.003750pt}%
\definecolor{currentstroke}{rgb}{0.000000,0.000000,0.000000}%
\pgfsetstrokecolor{currentstroke}%
\pgfsetdash{}{0pt}%
\pgfpathmoveto{\pgfqpoint{0.545444in}{0.489833in}}%
\pgfpathlineto{\pgfqpoint{0.545444in}{2.148139in}}%
\pgfusepath{stroke}%
\end{pgfscope}%
\begin{pgfscope}%
\pgfsetbuttcap%
\pgfsetroundjoin%
\definecolor{currentfill}{rgb}{0.000000,0.000000,0.000000}%
\pgfsetfillcolor{currentfill}%
\pgfsetlinewidth{0.501875pt}%
\definecolor{currentstroke}{rgb}{0.000000,0.000000,0.000000}%
\pgfsetstrokecolor{currentstroke}%
\pgfsetdash{}{0pt}%
\pgfsys@defobject{currentmarker}{\pgfqpoint{0.000000in}{0.000000in}}{\pgfqpoint{0.000000in}{0.055556in}}{%
\pgfpathmoveto{\pgfqpoint{0.000000in}{0.000000in}}%
\pgfpathlineto{\pgfqpoint{0.000000in}{0.055556in}}%
\pgfusepath{stroke,fill}%
}%
\begin{pgfscope}%
\pgfsys@transformshift{0.545444in}{0.489833in}%
\pgfsys@useobject{currentmarker}{}%
\end{pgfscope}%
\end{pgfscope}%
\begin{pgfscope}%
\pgfsetbuttcap%
\pgfsetroundjoin%
\definecolor{currentfill}{rgb}{0.000000,0.000000,0.000000}%
\pgfsetfillcolor{currentfill}%
\pgfsetlinewidth{0.501875pt}%
\definecolor{currentstroke}{rgb}{0.000000,0.000000,0.000000}%
\pgfsetstrokecolor{currentstroke}%
\pgfsetdash{}{0pt}%
\pgfsys@defobject{currentmarker}{\pgfqpoint{0.000000in}{-0.055556in}}{\pgfqpoint{0.000000in}{0.000000in}}{%
\pgfpathmoveto{\pgfqpoint{0.000000in}{0.000000in}}%
\pgfpathlineto{\pgfqpoint{0.000000in}{-0.055556in}}%
\pgfusepath{stroke,fill}%
}%
\begin{pgfscope}%
\pgfsys@transformshift{0.545444in}{2.148139in}%
\pgfsys@useobject{currentmarker}{}%
\end{pgfscope}%
\end{pgfscope}%
\begin{pgfscope}%
\pgftext[x=0.545444in,y=0.434277in,,top]{{\rmfamily\fontsize{11.000000}{13.200000}\selectfont 2.0}}%
\end{pgfscope}%
\begin{pgfscope}%
\pgfsetbuttcap%
\pgfsetroundjoin%
\definecolor{currentfill}{rgb}{0.000000,0.000000,0.000000}%
\pgfsetfillcolor{currentfill}%
\pgfsetlinewidth{0.501875pt}%
\definecolor{currentstroke}{rgb}{0.000000,0.000000,0.000000}%
\pgfsetstrokecolor{currentstroke}%
\pgfsetdash{}{0pt}%
\pgfsys@defobject{currentmarker}{\pgfqpoint{0.000000in}{0.000000in}}{\pgfqpoint{0.000000in}{0.055556in}}{%
\pgfpathmoveto{\pgfqpoint{0.000000in}{0.000000in}}%
\pgfpathlineto{\pgfqpoint{0.000000in}{0.055556in}}%
\pgfusepath{stroke,fill}%
}%
\begin{pgfscope}%
\pgfsys@transformshift{1.150990in}{0.489833in}%
\pgfsys@useobject{currentmarker}{}%
\end{pgfscope}%
\end{pgfscope}%
\begin{pgfscope}%
\pgfsetbuttcap%
\pgfsetroundjoin%
\definecolor{currentfill}{rgb}{0.000000,0.000000,0.000000}%
\pgfsetfillcolor{currentfill}%
\pgfsetlinewidth{0.501875pt}%
\definecolor{currentstroke}{rgb}{0.000000,0.000000,0.000000}%
\pgfsetstrokecolor{currentstroke}%
\pgfsetdash{}{0pt}%
\pgfsys@defobject{currentmarker}{\pgfqpoint{0.000000in}{-0.055556in}}{\pgfqpoint{0.000000in}{0.000000in}}{%
\pgfpathmoveto{\pgfqpoint{0.000000in}{0.000000in}}%
\pgfpathlineto{\pgfqpoint{0.000000in}{-0.055556in}}%
\pgfusepath{stroke,fill}%
}%
\begin{pgfscope}%
\pgfsys@transformshift{1.150990in}{2.148139in}%
\pgfsys@useobject{currentmarker}{}%
\end{pgfscope}%
\end{pgfscope}%
\begin{pgfscope}%
\pgftext[x=1.150990in,y=0.434277in,,top]{{\rmfamily\fontsize{11.000000}{13.200000}\selectfont 2.1}}%
\end{pgfscope}%
\begin{pgfscope}%
\pgfsetbuttcap%
\pgfsetroundjoin%
\definecolor{currentfill}{rgb}{0.000000,0.000000,0.000000}%
\pgfsetfillcolor{currentfill}%
\pgfsetlinewidth{0.501875pt}%
\definecolor{currentstroke}{rgb}{0.000000,0.000000,0.000000}%
\pgfsetstrokecolor{currentstroke}%
\pgfsetdash{}{0pt}%
\pgfsys@defobject{currentmarker}{\pgfqpoint{0.000000in}{0.000000in}}{\pgfqpoint{0.000000in}{0.055556in}}{%
\pgfpathmoveto{\pgfqpoint{0.000000in}{0.000000in}}%
\pgfpathlineto{\pgfqpoint{0.000000in}{0.055556in}}%
\pgfusepath{stroke,fill}%
}%
\begin{pgfscope}%
\pgfsys@transformshift{1.756535in}{0.489833in}%
\pgfsys@useobject{currentmarker}{}%
\end{pgfscope}%
\end{pgfscope}%
\begin{pgfscope}%
\pgfsetbuttcap%
\pgfsetroundjoin%
\definecolor{currentfill}{rgb}{0.000000,0.000000,0.000000}%
\pgfsetfillcolor{currentfill}%
\pgfsetlinewidth{0.501875pt}%
\definecolor{currentstroke}{rgb}{0.000000,0.000000,0.000000}%
\pgfsetstrokecolor{currentstroke}%
\pgfsetdash{}{0pt}%
\pgfsys@defobject{currentmarker}{\pgfqpoint{0.000000in}{-0.055556in}}{\pgfqpoint{0.000000in}{0.000000in}}{%
\pgfpathmoveto{\pgfqpoint{0.000000in}{0.000000in}}%
\pgfpathlineto{\pgfqpoint{0.000000in}{-0.055556in}}%
\pgfusepath{stroke,fill}%
}%
\begin{pgfscope}%
\pgfsys@transformshift{1.756535in}{2.148139in}%
\pgfsys@useobject{currentmarker}{}%
\end{pgfscope}%
\end{pgfscope}%
\begin{pgfscope}%
\pgftext[x=1.756535in,y=0.434277in,,top]{{\rmfamily\fontsize{11.000000}{13.200000}\selectfont 2.2}}%
\end{pgfscope}%
\begin{pgfscope}%
\pgfsetbuttcap%
\pgfsetroundjoin%
\definecolor{currentfill}{rgb}{0.000000,0.000000,0.000000}%
\pgfsetfillcolor{currentfill}%
\pgfsetlinewidth{0.501875pt}%
\definecolor{currentstroke}{rgb}{0.000000,0.000000,0.000000}%
\pgfsetstrokecolor{currentstroke}%
\pgfsetdash{}{0pt}%
\pgfsys@defobject{currentmarker}{\pgfqpoint{0.000000in}{0.000000in}}{\pgfqpoint{0.000000in}{0.055556in}}{%
\pgfpathmoveto{\pgfqpoint{0.000000in}{0.000000in}}%
\pgfpathlineto{\pgfqpoint{0.000000in}{0.055556in}}%
\pgfusepath{stroke,fill}%
}%
\begin{pgfscope}%
\pgfsys@transformshift{2.362080in}{0.489833in}%
\pgfsys@useobject{currentmarker}{}%
\end{pgfscope}%
\end{pgfscope}%
\begin{pgfscope}%
\pgfsetbuttcap%
\pgfsetroundjoin%
\definecolor{currentfill}{rgb}{0.000000,0.000000,0.000000}%
\pgfsetfillcolor{currentfill}%
\pgfsetlinewidth{0.501875pt}%
\definecolor{currentstroke}{rgb}{0.000000,0.000000,0.000000}%
\pgfsetstrokecolor{currentstroke}%
\pgfsetdash{}{0pt}%
\pgfsys@defobject{currentmarker}{\pgfqpoint{0.000000in}{-0.055556in}}{\pgfqpoint{0.000000in}{0.000000in}}{%
\pgfpathmoveto{\pgfqpoint{0.000000in}{0.000000in}}%
\pgfpathlineto{\pgfqpoint{0.000000in}{-0.055556in}}%
\pgfusepath{stroke,fill}%
}%
\begin{pgfscope}%
\pgfsys@transformshift{2.362080in}{2.148139in}%
\pgfsys@useobject{currentmarker}{}%
\end{pgfscope}%
\end{pgfscope}%
\begin{pgfscope}%
\pgftext[x=2.362080in,y=0.434277in,,top]{{\rmfamily\fontsize{11.000000}{13.200000}\selectfont 2.3}}%
\end{pgfscope}%
\begin{pgfscope}%
\pgfsetbuttcap%
\pgfsetroundjoin%
\definecolor{currentfill}{rgb}{0.000000,0.000000,0.000000}%
\pgfsetfillcolor{currentfill}%
\pgfsetlinewidth{0.501875pt}%
\definecolor{currentstroke}{rgb}{0.000000,0.000000,0.000000}%
\pgfsetstrokecolor{currentstroke}%
\pgfsetdash{}{0pt}%
\pgfsys@defobject{currentmarker}{\pgfqpoint{0.000000in}{0.000000in}}{\pgfqpoint{0.000000in}{0.055556in}}{%
\pgfpathmoveto{\pgfqpoint{0.000000in}{0.000000in}}%
\pgfpathlineto{\pgfqpoint{0.000000in}{0.055556in}}%
\pgfusepath{stroke,fill}%
}%
\begin{pgfscope}%
\pgfsys@transformshift{2.967626in}{0.489833in}%
\pgfsys@useobject{currentmarker}{}%
\end{pgfscope}%
\end{pgfscope}%
\begin{pgfscope}%
\pgfsetbuttcap%
\pgfsetroundjoin%
\definecolor{currentfill}{rgb}{0.000000,0.000000,0.000000}%
\pgfsetfillcolor{currentfill}%
\pgfsetlinewidth{0.501875pt}%
\definecolor{currentstroke}{rgb}{0.000000,0.000000,0.000000}%
\pgfsetstrokecolor{currentstroke}%
\pgfsetdash{}{0pt}%
\pgfsys@defobject{currentmarker}{\pgfqpoint{0.000000in}{-0.055556in}}{\pgfqpoint{0.000000in}{0.000000in}}{%
\pgfpathmoveto{\pgfqpoint{0.000000in}{0.000000in}}%
\pgfpathlineto{\pgfqpoint{0.000000in}{-0.055556in}}%
\pgfusepath{stroke,fill}%
}%
\begin{pgfscope}%
\pgfsys@transformshift{2.967626in}{2.148139in}%
\pgfsys@useobject{currentmarker}{}%
\end{pgfscope}%
\end{pgfscope}%
\begin{pgfscope}%
\pgftext[x=2.967626in,y=0.434277in,,top]{{\rmfamily\fontsize{11.000000}{13.200000}\selectfont 2.4}}%
\end{pgfscope}%
\begin{pgfscope}%
\pgftext[x=1.756535in,y=0.229166in,,top]{{\rmfamily\fontsize{11.000000}{13.200000}\selectfont \(\displaystyle k_bT\)\quad[ \(\displaystyle J\) ]}}%
\end{pgfscope}%
\begin{pgfscope}%
\pgfsetbuttcap%
\pgfsetroundjoin%
\definecolor{currentfill}{rgb}{0.000000,0.000000,0.000000}%
\pgfsetfillcolor{currentfill}%
\pgfsetlinewidth{0.501875pt}%
\definecolor{currentstroke}{rgb}{0.000000,0.000000,0.000000}%
\pgfsetstrokecolor{currentstroke}%
\pgfsetdash{}{0pt}%
\pgfsys@defobject{currentmarker}{\pgfqpoint{0.000000in}{0.000000in}}{\pgfqpoint{0.055556in}{0.000000in}}{%
\pgfpathmoveto{\pgfqpoint{0.000000in}{0.000000in}}%
\pgfpathlineto{\pgfqpoint{0.055556in}{0.000000in}}%
\pgfusepath{stroke,fill}%
}%
\begin{pgfscope}%
\pgfsys@transformshift{0.545444in}{0.489833in}%
\pgfsys@useobject{currentmarker}{}%
\end{pgfscope}%
\end{pgfscope}%
\begin{pgfscope}%
\pgfsetbuttcap%
\pgfsetroundjoin%
\definecolor{currentfill}{rgb}{0.000000,0.000000,0.000000}%
\pgfsetfillcolor{currentfill}%
\pgfsetlinewidth{0.501875pt}%
\definecolor{currentstroke}{rgb}{0.000000,0.000000,0.000000}%
\pgfsetstrokecolor{currentstroke}%
\pgfsetdash{}{0pt}%
\pgfsys@defobject{currentmarker}{\pgfqpoint{-0.055556in}{0.000000in}}{\pgfqpoint{0.000000in}{0.000000in}}{%
\pgfpathmoveto{\pgfqpoint{0.000000in}{0.000000in}}%
\pgfpathlineto{\pgfqpoint{-0.055556in}{0.000000in}}%
\pgfusepath{stroke,fill}%
}%
\begin{pgfscope}%
\pgfsys@transformshift{2.967626in}{0.489833in}%
\pgfsys@useobject{currentmarker}{}%
\end{pgfscope}%
\end{pgfscope}%
\begin{pgfscope}%
\pgftext[x=0.489889in,y=0.489833in,right,]{{\rmfamily\fontsize{11.000000}{13.200000}\selectfont 0.5}}%
\end{pgfscope}%
\begin{pgfscope}%
\pgfsetbuttcap%
\pgfsetroundjoin%
\definecolor{currentfill}{rgb}{0.000000,0.000000,0.000000}%
\pgfsetfillcolor{currentfill}%
\pgfsetlinewidth{0.501875pt}%
\definecolor{currentstroke}{rgb}{0.000000,0.000000,0.000000}%
\pgfsetstrokecolor{currentstroke}%
\pgfsetdash{}{0pt}%
\pgfsys@defobject{currentmarker}{\pgfqpoint{0.000000in}{0.000000in}}{\pgfqpoint{0.055556in}{0.000000in}}{%
\pgfpathmoveto{\pgfqpoint{0.000000in}{0.000000in}}%
\pgfpathlineto{\pgfqpoint{0.055556in}{0.000000in}}%
\pgfusepath{stroke,fill}%
}%
\begin{pgfscope}%
\pgfsys@transformshift{0.545444in}{0.821494in}%
\pgfsys@useobject{currentmarker}{}%
\end{pgfscope}%
\end{pgfscope}%
\begin{pgfscope}%
\pgfsetbuttcap%
\pgfsetroundjoin%
\definecolor{currentfill}{rgb}{0.000000,0.000000,0.000000}%
\pgfsetfillcolor{currentfill}%
\pgfsetlinewidth{0.501875pt}%
\definecolor{currentstroke}{rgb}{0.000000,0.000000,0.000000}%
\pgfsetstrokecolor{currentstroke}%
\pgfsetdash{}{0pt}%
\pgfsys@defobject{currentmarker}{\pgfqpoint{-0.055556in}{0.000000in}}{\pgfqpoint{0.000000in}{0.000000in}}{%
\pgfpathmoveto{\pgfqpoint{0.000000in}{0.000000in}}%
\pgfpathlineto{\pgfqpoint{-0.055556in}{0.000000in}}%
\pgfusepath{stroke,fill}%
}%
\begin{pgfscope}%
\pgfsys@transformshift{2.967626in}{0.821494in}%
\pgfsys@useobject{currentmarker}{}%
\end{pgfscope}%
\end{pgfscope}%
\begin{pgfscope}%
\pgftext[x=0.489889in,y=0.821494in,right,]{{\rmfamily\fontsize{11.000000}{13.200000}\selectfont 1.0}}%
\end{pgfscope}%
\begin{pgfscope}%
\pgfsetbuttcap%
\pgfsetroundjoin%
\definecolor{currentfill}{rgb}{0.000000,0.000000,0.000000}%
\pgfsetfillcolor{currentfill}%
\pgfsetlinewidth{0.501875pt}%
\definecolor{currentstroke}{rgb}{0.000000,0.000000,0.000000}%
\pgfsetstrokecolor{currentstroke}%
\pgfsetdash{}{0pt}%
\pgfsys@defobject{currentmarker}{\pgfqpoint{0.000000in}{0.000000in}}{\pgfqpoint{0.055556in}{0.000000in}}{%
\pgfpathmoveto{\pgfqpoint{0.000000in}{0.000000in}}%
\pgfpathlineto{\pgfqpoint{0.055556in}{0.000000in}}%
\pgfusepath{stroke,fill}%
}%
\begin{pgfscope}%
\pgfsys@transformshift{0.545444in}{1.153155in}%
\pgfsys@useobject{currentmarker}{}%
\end{pgfscope}%
\end{pgfscope}%
\begin{pgfscope}%
\pgfsetbuttcap%
\pgfsetroundjoin%
\definecolor{currentfill}{rgb}{0.000000,0.000000,0.000000}%
\pgfsetfillcolor{currentfill}%
\pgfsetlinewidth{0.501875pt}%
\definecolor{currentstroke}{rgb}{0.000000,0.000000,0.000000}%
\pgfsetstrokecolor{currentstroke}%
\pgfsetdash{}{0pt}%
\pgfsys@defobject{currentmarker}{\pgfqpoint{-0.055556in}{0.000000in}}{\pgfqpoint{0.000000in}{0.000000in}}{%
\pgfpathmoveto{\pgfqpoint{0.000000in}{0.000000in}}%
\pgfpathlineto{\pgfqpoint{-0.055556in}{0.000000in}}%
\pgfusepath{stroke,fill}%
}%
\begin{pgfscope}%
\pgfsys@transformshift{2.967626in}{1.153155in}%
\pgfsys@useobject{currentmarker}{}%
\end{pgfscope}%
\end{pgfscope}%
\begin{pgfscope}%
\pgftext[x=0.489889in,y=1.153155in,right,]{{\rmfamily\fontsize{11.000000}{13.200000}\selectfont 1.5}}%
\end{pgfscope}%
\begin{pgfscope}%
\pgfsetbuttcap%
\pgfsetroundjoin%
\definecolor{currentfill}{rgb}{0.000000,0.000000,0.000000}%
\pgfsetfillcolor{currentfill}%
\pgfsetlinewidth{0.501875pt}%
\definecolor{currentstroke}{rgb}{0.000000,0.000000,0.000000}%
\pgfsetstrokecolor{currentstroke}%
\pgfsetdash{}{0pt}%
\pgfsys@defobject{currentmarker}{\pgfqpoint{0.000000in}{0.000000in}}{\pgfqpoint{0.055556in}{0.000000in}}{%
\pgfpathmoveto{\pgfqpoint{0.000000in}{0.000000in}}%
\pgfpathlineto{\pgfqpoint{0.055556in}{0.000000in}}%
\pgfusepath{stroke,fill}%
}%
\begin{pgfscope}%
\pgfsys@transformshift{0.545444in}{1.484816in}%
\pgfsys@useobject{currentmarker}{}%
\end{pgfscope}%
\end{pgfscope}%
\begin{pgfscope}%
\pgfsetbuttcap%
\pgfsetroundjoin%
\definecolor{currentfill}{rgb}{0.000000,0.000000,0.000000}%
\pgfsetfillcolor{currentfill}%
\pgfsetlinewidth{0.501875pt}%
\definecolor{currentstroke}{rgb}{0.000000,0.000000,0.000000}%
\pgfsetstrokecolor{currentstroke}%
\pgfsetdash{}{0pt}%
\pgfsys@defobject{currentmarker}{\pgfqpoint{-0.055556in}{0.000000in}}{\pgfqpoint{0.000000in}{0.000000in}}{%
\pgfpathmoveto{\pgfqpoint{0.000000in}{0.000000in}}%
\pgfpathlineto{\pgfqpoint{-0.055556in}{0.000000in}}%
\pgfusepath{stroke,fill}%
}%
\begin{pgfscope}%
\pgfsys@transformshift{2.967626in}{1.484816in}%
\pgfsys@useobject{currentmarker}{}%
\end{pgfscope}%
\end{pgfscope}%
\begin{pgfscope}%
\pgftext[x=0.489889in,y=1.484816in,right,]{{\rmfamily\fontsize{11.000000}{13.200000}\selectfont 2.0}}%
\end{pgfscope}%
\begin{pgfscope}%
\pgfsetbuttcap%
\pgfsetroundjoin%
\definecolor{currentfill}{rgb}{0.000000,0.000000,0.000000}%
\pgfsetfillcolor{currentfill}%
\pgfsetlinewidth{0.501875pt}%
\definecolor{currentstroke}{rgb}{0.000000,0.000000,0.000000}%
\pgfsetstrokecolor{currentstroke}%
\pgfsetdash{}{0pt}%
\pgfsys@defobject{currentmarker}{\pgfqpoint{0.000000in}{0.000000in}}{\pgfqpoint{0.055556in}{0.000000in}}{%
\pgfpathmoveto{\pgfqpoint{0.000000in}{0.000000in}}%
\pgfpathlineto{\pgfqpoint{0.055556in}{0.000000in}}%
\pgfusepath{stroke,fill}%
}%
\begin{pgfscope}%
\pgfsys@transformshift{0.545444in}{1.816478in}%
\pgfsys@useobject{currentmarker}{}%
\end{pgfscope}%
\end{pgfscope}%
\begin{pgfscope}%
\pgfsetbuttcap%
\pgfsetroundjoin%
\definecolor{currentfill}{rgb}{0.000000,0.000000,0.000000}%
\pgfsetfillcolor{currentfill}%
\pgfsetlinewidth{0.501875pt}%
\definecolor{currentstroke}{rgb}{0.000000,0.000000,0.000000}%
\pgfsetstrokecolor{currentstroke}%
\pgfsetdash{}{0pt}%
\pgfsys@defobject{currentmarker}{\pgfqpoint{-0.055556in}{0.000000in}}{\pgfqpoint{0.000000in}{0.000000in}}{%
\pgfpathmoveto{\pgfqpoint{0.000000in}{0.000000in}}%
\pgfpathlineto{\pgfqpoint{-0.055556in}{0.000000in}}%
\pgfusepath{stroke,fill}%
}%
\begin{pgfscope}%
\pgfsys@transformshift{2.967626in}{1.816478in}%
\pgfsys@useobject{currentmarker}{}%
\end{pgfscope}%
\end{pgfscope}%
\begin{pgfscope}%
\pgftext[x=0.489889in,y=1.816478in,right,]{{\rmfamily\fontsize{11.000000}{13.200000}\selectfont 2.5}}%
\end{pgfscope}%
\begin{pgfscope}%
\pgfsetbuttcap%
\pgfsetroundjoin%
\definecolor{currentfill}{rgb}{0.000000,0.000000,0.000000}%
\pgfsetfillcolor{currentfill}%
\pgfsetlinewidth{0.501875pt}%
\definecolor{currentstroke}{rgb}{0.000000,0.000000,0.000000}%
\pgfsetstrokecolor{currentstroke}%
\pgfsetdash{}{0pt}%
\pgfsys@defobject{currentmarker}{\pgfqpoint{0.000000in}{0.000000in}}{\pgfqpoint{0.055556in}{0.000000in}}{%
\pgfpathmoveto{\pgfqpoint{0.000000in}{0.000000in}}%
\pgfpathlineto{\pgfqpoint{0.055556in}{0.000000in}}%
\pgfusepath{stroke,fill}%
}%
\begin{pgfscope}%
\pgfsys@transformshift{0.545444in}{2.148139in}%
\pgfsys@useobject{currentmarker}{}%
\end{pgfscope}%
\end{pgfscope}%
\begin{pgfscope}%
\pgfsetbuttcap%
\pgfsetroundjoin%
\definecolor{currentfill}{rgb}{0.000000,0.000000,0.000000}%
\pgfsetfillcolor{currentfill}%
\pgfsetlinewidth{0.501875pt}%
\definecolor{currentstroke}{rgb}{0.000000,0.000000,0.000000}%
\pgfsetstrokecolor{currentstroke}%
\pgfsetdash{}{0pt}%
\pgfsys@defobject{currentmarker}{\pgfqpoint{-0.055556in}{0.000000in}}{\pgfqpoint{0.000000in}{0.000000in}}{%
\pgfpathmoveto{\pgfqpoint{0.000000in}{0.000000in}}%
\pgfpathlineto{\pgfqpoint{-0.055556in}{0.000000in}}%
\pgfusepath{stroke,fill}%
}%
\begin{pgfscope}%
\pgfsys@transformshift{2.967626in}{2.148139in}%
\pgfsys@useobject{currentmarker}{}%
\end{pgfscope}%
\end{pgfscope}%
\begin{pgfscope}%
\pgftext[x=0.489889in,y=2.148139in,right,]{{\rmfamily\fontsize{11.000000}{13.200000}\selectfont 3.0}}%
\end{pgfscope}%
\begin{pgfscope}%
\pgftext[x=0.229166in,y=1.318986in,,bottom,rotate=90.000000]{{\rmfamily\fontsize{11.000000}{13.200000}\selectfont \(\displaystyle \chi\)\quad[ \(\displaystyle \cdot\) ]}}%
\end{pgfscope}%
\begin{pgfscope}%
\pgftext[x=1.756535in,y=2.217583in,,base]{{\rmfamily\fontsize{11.000000}{13.200000}\selectfont Magnetic succeptability vs thermal energy}}%
\end{pgfscope}%
\begin{pgfscope}%
\pgfsetbuttcap%
\pgfsetmiterjoin%
\definecolor{currentfill}{rgb}{1.000000,1.000000,1.000000}%
\pgfsetfillcolor{currentfill}%
\pgfsetlinewidth{0.000000pt}%
\definecolor{currentstroke}{rgb}{0.000000,0.000000,0.000000}%
\pgfsetstrokecolor{currentstroke}%
\pgfsetstrokeopacity{0.000000}%
\pgfsetdash{}{0pt}%
\pgfpathmoveto{\pgfqpoint{3.805790in}{0.489833in}}%
\pgfpathlineto{\pgfqpoint{6.227972in}{0.489833in}}%
\pgfpathlineto{\pgfqpoint{6.227972in}{2.148139in}}%
\pgfpathlineto{\pgfqpoint{3.805790in}{2.148139in}}%
\pgfpathclose%
\pgfusepath{fill}%
\end{pgfscope}%
\begin{pgfscope}%
\pgfpathrectangle{\pgfqpoint{3.805790in}{0.489833in}}{\pgfqpoint{2.422182in}{1.658306in}} %
\pgfusepath{clip}%
\pgfsetbuttcap%
\pgfsetroundjoin%
\pgfsetlinewidth{1.003750pt}%
\definecolor{currentstroke}{rgb}{0.000000,0.000000,0.000000}%
\pgfsetstrokecolor{currentstroke}%
\pgfsetdash{{8.000000pt}{4.000000pt}{2.000000pt}{4.000000pt}{2.000000pt}{4.000000pt}}{0.000000pt}%
\pgfpathmoveto{\pgfqpoint{3.805790in}{0.491948in}}%
\pgfpathlineto{\pgfqpoint{4.399225in}{0.494997in}}%
\pgfpathlineto{\pgfqpoint{4.689887in}{0.498680in}}%
\pgfpathlineto{\pgfqpoint{4.919994in}{0.503716in}}%
\pgfpathlineto{\pgfqpoint{5.125880in}{0.510403in}}%
\pgfpathlineto{\pgfqpoint{5.440763in}{0.523576in}}%
\pgfpathlineto{\pgfqpoint{5.719314in}{0.534546in}}%
\pgfpathlineto{\pgfqpoint{5.876756in}{0.538169in}}%
\pgfpathlineto{\pgfqpoint{6.034198in}{0.539722in}}%
\pgfpathlineto{\pgfqpoint{6.227972in}{0.539114in}}%
\pgfpathlineto{\pgfqpoint{6.227972in}{0.539114in}}%
\pgfusepath{stroke}%
\end{pgfscope}%
\begin{pgfscope}%
\pgfpathrectangle{\pgfqpoint{3.805790in}{0.489833in}}{\pgfqpoint{2.422182in}{1.658306in}} %
\pgfusepath{clip}%
\pgfsetbuttcap%
\pgfsetroundjoin%
\pgfsetlinewidth{1.003750pt}%
\definecolor{currentstroke}{rgb}{0.000000,0.000000,0.000000}%
\pgfsetstrokecolor{currentstroke}%
\pgfsetdash{{6.000000pt}{6.000000pt}}{0.000000pt}%
\pgfpathmoveto{\pgfqpoint{3.805790in}{0.491880in}}%
\pgfpathlineto{\pgfqpoint{4.447669in}{0.495377in}}%
\pgfpathlineto{\pgfqpoint{4.665665in}{0.498702in}}%
\pgfpathlineto{\pgfqpoint{4.810996in}{0.503061in}}%
\pgfpathlineto{\pgfqpoint{4.919994in}{0.508323in}}%
\pgfpathlineto{\pgfqpoint{4.992660in}{0.513796in}}%
\pgfpathlineto{\pgfqpoint{5.041103in}{0.519533in}}%
\pgfpathlineto{\pgfqpoint{5.089547in}{0.527789in}}%
\pgfpathlineto{\pgfqpoint{5.125880in}{0.536072in}}%
\pgfpathlineto{\pgfqpoint{5.162212in}{0.546491in}}%
\pgfpathlineto{\pgfqpoint{5.198545in}{0.559369in}}%
\pgfpathlineto{\pgfqpoint{5.234878in}{0.575037in}}%
\pgfpathlineto{\pgfqpoint{5.271210in}{0.593829in}}%
\pgfpathlineto{\pgfqpoint{5.307543in}{0.616156in}}%
\pgfpathlineto{\pgfqpoint{5.331765in}{0.633268in}}%
\pgfpathlineto{\pgfqpoint{5.355987in}{0.652343in}}%
\pgfpathlineto{\pgfqpoint{5.392320in}{0.683982in}}%
\pgfpathlineto{\pgfqpoint{5.464985in}{0.748999in}}%
\pgfpathlineto{\pgfqpoint{5.501318in}{0.778480in}}%
\pgfpathlineto{\pgfqpoint{5.525540in}{0.795947in}}%
\pgfpathlineto{\pgfqpoint{5.549761in}{0.810907in}}%
\pgfpathlineto{\pgfqpoint{5.573983in}{0.822604in}}%
\pgfpathlineto{\pgfqpoint{5.598205in}{0.830676in}}%
\pgfpathlineto{\pgfqpoint{5.622427in}{0.835101in}}%
\pgfpathlineto{\pgfqpoint{5.646649in}{0.836017in}}%
\pgfpathlineto{\pgfqpoint{5.670870in}{0.833728in}}%
\pgfpathlineto{\pgfqpoint{5.695092in}{0.828583in}}%
\pgfpathlineto{\pgfqpoint{5.719314in}{0.821065in}}%
\pgfpathlineto{\pgfqpoint{5.755647in}{0.806438in}}%
\pgfpathlineto{\pgfqpoint{5.791980in}{0.789069in}}%
\pgfpathlineto{\pgfqpoint{5.985754in}{0.691291in}}%
\pgfpathlineto{\pgfqpoint{6.034198in}{0.670600in}}%
\pgfpathlineto{\pgfqpoint{6.082641in}{0.652608in}}%
\pgfpathlineto{\pgfqpoint{6.131085in}{0.636877in}}%
\pgfpathlineto{\pgfqpoint{6.203750in}{0.615942in}}%
\pgfpathlineto{\pgfqpoint{6.227972in}{0.609257in}}%
\pgfpathlineto{\pgfqpoint{6.227972in}{0.609257in}}%
\pgfusepath{stroke}%
\end{pgfscope}%
\begin{pgfscope}%
\pgfpathrectangle{\pgfqpoint{3.805790in}{0.489833in}}{\pgfqpoint{2.422182in}{1.658306in}} %
\pgfusepath{clip}%
\pgfsetbuttcap%
\pgfsetroundjoin%
\pgfsetlinewidth{1.003750pt}%
\definecolor{currentstroke}{rgb}{0.000000,0.000000,0.000000}%
\pgfsetstrokecolor{currentstroke}%
\pgfsetdash{{2.000000pt}{2.000000pt}}{0.000000pt}%
\pgfpathmoveto{\pgfqpoint{3.805790in}{0.491832in}}%
\pgfpathlineto{\pgfqpoint{4.496112in}{0.495833in}}%
\pgfpathlineto{\pgfqpoint{4.677776in}{0.499319in}}%
\pgfpathlineto{\pgfqpoint{4.835218in}{0.504486in}}%
\pgfpathlineto{\pgfqpoint{4.956327in}{0.510408in}}%
\pgfpathlineto{\pgfqpoint{5.004770in}{0.514506in}}%
\pgfpathlineto{\pgfqpoint{5.053214in}{0.521012in}}%
\pgfpathlineto{\pgfqpoint{5.089547in}{0.528082in}}%
\pgfpathlineto{\pgfqpoint{5.125880in}{0.537503in}}%
\pgfpathlineto{\pgfqpoint{5.162212in}{0.549821in}}%
\pgfpathlineto{\pgfqpoint{5.186434in}{0.560408in}}%
\pgfpathlineto{\pgfqpoint{5.210656in}{0.573657in}}%
\pgfpathlineto{\pgfqpoint{5.234878in}{0.590223in}}%
\pgfpathlineto{\pgfqpoint{5.259100in}{0.610762in}}%
\pgfpathlineto{\pgfqpoint{5.283321in}{0.636298in}}%
\pgfpathlineto{\pgfqpoint{5.295432in}{0.651708in}}%
\pgfpathlineto{\pgfqpoint{5.307543in}{0.669367in}}%
\pgfpathlineto{\pgfqpoint{5.319654in}{0.689641in}}%
\pgfpathlineto{\pgfqpoint{5.331765in}{0.712895in}}%
\pgfpathlineto{\pgfqpoint{5.343876in}{0.739495in}}%
\pgfpathlineto{\pgfqpoint{5.355987in}{0.769782in}}%
\pgfpathlineto{\pgfqpoint{5.368098in}{0.803849in}}%
\pgfpathlineto{\pgfqpoint{5.380209in}{0.841640in}}%
\pgfpathlineto{\pgfqpoint{5.404430in}{0.928165in}}%
\pgfpathlineto{\pgfqpoint{5.464985in}{1.168285in}}%
\pgfpathlineto{\pgfqpoint{5.477096in}{1.208894in}}%
\pgfpathlineto{\pgfqpoint{5.489207in}{1.244768in}}%
\pgfpathlineto{\pgfqpoint{5.501318in}{1.275241in}}%
\pgfpathlineto{\pgfqpoint{5.513429in}{1.299679in}}%
\pgfpathlineto{\pgfqpoint{5.525540in}{1.317910in}}%
\pgfpathlineto{\pgfqpoint{5.537650in}{1.330119in}}%
\pgfpathlineto{\pgfqpoint{5.549761in}{1.336500in}}%
\pgfpathlineto{\pgfqpoint{5.561872in}{1.337250in}}%
\pgfpathlineto{\pgfqpoint{5.573983in}{1.332655in}}%
\pgfpathlineto{\pgfqpoint{5.586094in}{1.323260in}}%
\pgfpathlineto{\pgfqpoint{5.598205in}{1.309648in}}%
\pgfpathlineto{\pgfqpoint{5.610316in}{1.292407in}}%
\pgfpathlineto{\pgfqpoint{5.622427in}{1.272116in}}%
\pgfpathlineto{\pgfqpoint{5.646649in}{1.224598in}}%
\pgfpathlineto{\pgfqpoint{5.682981in}{1.144138in}}%
\pgfpathlineto{\pgfqpoint{5.719314in}{1.063648in}}%
\pgfpathlineto{\pgfqpoint{5.743536in}{1.014624in}}%
\pgfpathlineto{\pgfqpoint{5.767758in}{0.970619in}}%
\pgfpathlineto{\pgfqpoint{5.791980in}{0.931112in}}%
\pgfpathlineto{\pgfqpoint{5.816201in}{0.895520in}}%
\pgfpathlineto{\pgfqpoint{5.840423in}{0.863102in}}%
\pgfpathlineto{\pgfqpoint{5.876756in}{0.818756in}}%
\pgfpathlineto{\pgfqpoint{5.900978in}{0.791856in}}%
\pgfpathlineto{\pgfqpoint{5.925200in}{0.767706in}}%
\pgfpathlineto{\pgfqpoint{5.949421in}{0.746649in}}%
\pgfpathlineto{\pgfqpoint{5.973643in}{0.728360in}}%
\pgfpathlineto{\pgfqpoint{5.997865in}{0.712392in}}%
\pgfpathlineto{\pgfqpoint{6.034198in}{0.691822in}}%
\pgfpathlineto{\pgfqpoint{6.070530in}{0.674178in}}%
\pgfpathlineto{\pgfqpoint{6.106863in}{0.658916in}}%
\pgfpathlineto{\pgfqpoint{6.155307in}{0.641617in}}%
\pgfpathlineto{\pgfqpoint{6.215861in}{0.622957in}}%
\pgfpathlineto{\pgfqpoint{6.227972in}{0.619362in}}%
\pgfpathlineto{\pgfqpoint{6.227972in}{0.619362in}}%
\pgfusepath{stroke}%
\end{pgfscope}%
\begin{pgfscope}%
\pgfpathrectangle{\pgfqpoint{3.805790in}{0.489833in}}{\pgfqpoint{2.422182in}{1.658306in}} %
\pgfusepath{clip}%
\pgfsetrectcap%
\pgfsetroundjoin%
\pgfsetlinewidth{1.003750pt}%
\definecolor{currentstroke}{rgb}{0.000000,0.000000,0.000000}%
\pgfsetstrokecolor{currentstroke}%
\pgfsetdash{}{0pt}%
\pgfpathmoveto{\pgfqpoint{3.805790in}{0.492195in}}%
\pgfpathlineto{\pgfqpoint{4.205450in}{0.493407in}}%
\pgfpathlineto{\pgfqpoint{4.653554in}{0.498181in}}%
\pgfpathlineto{\pgfqpoint{4.774663in}{0.500805in}}%
\pgfpathlineto{\pgfqpoint{4.871550in}{0.505118in}}%
\pgfpathlineto{\pgfqpoint{4.944216in}{0.510301in}}%
\pgfpathlineto{\pgfqpoint{5.004770in}{0.516467in}}%
\pgfpathlineto{\pgfqpoint{5.065325in}{0.524944in}}%
\pgfpathlineto{\pgfqpoint{5.113769in}{0.533792in}}%
\pgfpathlineto{\pgfqpoint{5.150101in}{0.541900in}}%
\pgfpathlineto{\pgfqpoint{5.174323in}{0.549208in}}%
\pgfpathlineto{\pgfqpoint{5.198545in}{0.560141in}}%
\pgfpathlineto{\pgfqpoint{5.210656in}{0.567562in}}%
\pgfpathlineto{\pgfqpoint{5.222767in}{0.576605in}}%
\pgfpathlineto{\pgfqpoint{5.234878in}{0.587507in}}%
\pgfpathlineto{\pgfqpoint{5.246989in}{0.600507in}}%
\pgfpathlineto{\pgfqpoint{5.259100in}{0.615843in}}%
\pgfpathlineto{\pgfqpoint{5.271210in}{0.633800in}}%
\pgfpathlineto{\pgfqpoint{5.283321in}{0.654857in}}%
\pgfpathlineto{\pgfqpoint{5.295432in}{0.679547in}}%
\pgfpathlineto{\pgfqpoint{5.307543in}{0.708400in}}%
\pgfpathlineto{\pgfqpoint{5.319654in}{0.741947in}}%
\pgfpathlineto{\pgfqpoint{5.331765in}{0.780720in}}%
\pgfpathlineto{\pgfqpoint{5.343876in}{0.825249in}}%
\pgfpathlineto{\pgfqpoint{5.355987in}{0.876138in}}%
\pgfpathlineto{\pgfqpoint{5.368098in}{0.934727in}}%
\pgfpathlineto{\pgfqpoint{5.380209in}{1.002799in}}%
\pgfpathlineto{\pgfqpoint{5.392320in}{1.082142in}}%
\pgfpathlineto{\pgfqpoint{5.404430in}{1.174532in}}%
\pgfpathlineto{\pgfqpoint{5.428652in}{1.392629in}}%
\pgfpathlineto{\pgfqpoint{5.452874in}{1.620168in}}%
\pgfpathlineto{\pgfqpoint{5.464985in}{1.724086in}}%
\pgfpathlineto{\pgfqpoint{5.477096in}{1.815957in}}%
\pgfpathlineto{\pgfqpoint{5.489207in}{1.892394in}}%
\pgfpathlineto{\pgfqpoint{5.501318in}{1.950011in}}%
\pgfpathlineto{\pgfqpoint{5.513429in}{1.985625in}}%
\pgfpathlineto{\pgfqpoint{5.525540in}{1.999125in}}%
\pgfpathlineto{\pgfqpoint{5.537650in}{1.992771in}}%
\pgfpathlineto{\pgfqpoint{5.549761in}{1.968888in}}%
\pgfpathlineto{\pgfqpoint{5.561872in}{1.929799in}}%
\pgfpathlineto{\pgfqpoint{5.573983in}{1.878085in}}%
\pgfpathlineto{\pgfqpoint{5.586094in}{1.817030in}}%
\pgfpathlineto{\pgfqpoint{5.658760in}{1.418029in}}%
\pgfpathlineto{\pgfqpoint{5.682981in}{1.307247in}}%
\pgfpathlineto{\pgfqpoint{5.707203in}{1.213107in}}%
\pgfpathlineto{\pgfqpoint{5.731425in}{1.133059in}}%
\pgfpathlineto{\pgfqpoint{5.755647in}{1.064582in}}%
\pgfpathlineto{\pgfqpoint{5.779869in}{1.005629in}}%
\pgfpathlineto{\pgfqpoint{5.804090in}{0.954249in}}%
\pgfpathlineto{\pgfqpoint{5.828312in}{0.909060in}}%
\pgfpathlineto{\pgfqpoint{5.852534in}{0.869550in}}%
\pgfpathlineto{\pgfqpoint{5.876756in}{0.835258in}}%
\pgfpathlineto{\pgfqpoint{5.900978in}{0.805538in}}%
\pgfpathlineto{\pgfqpoint{5.925200in}{0.779646in}}%
\pgfpathlineto{\pgfqpoint{5.949421in}{0.756923in}}%
\pgfpathlineto{\pgfqpoint{5.973643in}{0.736967in}}%
\pgfpathlineto{\pgfqpoint{5.997865in}{0.719419in}}%
\pgfpathlineto{\pgfqpoint{6.022087in}{0.703921in}}%
\pgfpathlineto{\pgfqpoint{6.058420in}{0.683762in}}%
\pgfpathlineto{\pgfqpoint{6.094752in}{0.666578in}}%
\pgfpathlineto{\pgfqpoint{6.131085in}{0.651703in}}%
\pgfpathlineto{\pgfqpoint{6.179529in}{0.634315in}}%
\pgfpathlineto{\pgfqpoint{6.227972in}{0.618275in}}%
\pgfpathlineto{\pgfqpoint{6.227972in}{0.618275in}}%
\pgfusepath{stroke}%
\end{pgfscope}%
\begin{pgfscope}%
\pgfsetrectcap%
\pgfsetmiterjoin%
\pgfsetlinewidth{1.003750pt}%
\definecolor{currentstroke}{rgb}{0.000000,0.000000,0.000000}%
\pgfsetstrokecolor{currentstroke}%
\pgfsetdash{}{0pt}%
\pgfpathmoveto{\pgfqpoint{3.805790in}{2.148139in}}%
\pgfpathlineto{\pgfqpoint{6.227972in}{2.148139in}}%
\pgfusepath{stroke}%
\end{pgfscope}%
\begin{pgfscope}%
\pgfsetrectcap%
\pgfsetmiterjoin%
\pgfsetlinewidth{1.003750pt}%
\definecolor{currentstroke}{rgb}{0.000000,0.000000,0.000000}%
\pgfsetstrokecolor{currentstroke}%
\pgfsetdash{}{0pt}%
\pgfpathmoveto{\pgfqpoint{6.227972in}{0.489833in}}%
\pgfpathlineto{\pgfqpoint{6.227972in}{2.148139in}}%
\pgfusepath{stroke}%
\end{pgfscope}%
\begin{pgfscope}%
\pgfsetrectcap%
\pgfsetmiterjoin%
\pgfsetlinewidth{1.003750pt}%
\definecolor{currentstroke}{rgb}{0.000000,0.000000,0.000000}%
\pgfsetstrokecolor{currentstroke}%
\pgfsetdash{}{0pt}%
\pgfpathmoveto{\pgfqpoint{3.805790in}{0.489833in}}%
\pgfpathlineto{\pgfqpoint{6.227972in}{0.489833in}}%
\pgfusepath{stroke}%
\end{pgfscope}%
\begin{pgfscope}%
\pgfsetrectcap%
\pgfsetmiterjoin%
\pgfsetlinewidth{1.003750pt}%
\definecolor{currentstroke}{rgb}{0.000000,0.000000,0.000000}%
\pgfsetstrokecolor{currentstroke}%
\pgfsetdash{}{0pt}%
\pgfpathmoveto{\pgfqpoint{3.805790in}{0.489833in}}%
\pgfpathlineto{\pgfqpoint{3.805790in}{2.148139in}}%
\pgfusepath{stroke}%
\end{pgfscope}%
\begin{pgfscope}%
\pgfsetbuttcap%
\pgfsetroundjoin%
\definecolor{currentfill}{rgb}{0.000000,0.000000,0.000000}%
\pgfsetfillcolor{currentfill}%
\pgfsetlinewidth{0.501875pt}%
\definecolor{currentstroke}{rgb}{0.000000,0.000000,0.000000}%
\pgfsetstrokecolor{currentstroke}%
\pgfsetdash{}{0pt}%
\pgfsys@defobject{currentmarker}{\pgfqpoint{0.000000in}{0.000000in}}{\pgfqpoint{0.000000in}{0.055556in}}{%
\pgfpathmoveto{\pgfqpoint{0.000000in}{0.000000in}}%
\pgfpathlineto{\pgfqpoint{0.000000in}{0.055556in}}%
\pgfusepath{stroke,fill}%
}%
\begin{pgfscope}%
\pgfsys@transformshift{3.805790in}{0.489833in}%
\pgfsys@useobject{currentmarker}{}%
\end{pgfscope}%
\end{pgfscope}%
\begin{pgfscope}%
\pgfsetbuttcap%
\pgfsetroundjoin%
\definecolor{currentfill}{rgb}{0.000000,0.000000,0.000000}%
\pgfsetfillcolor{currentfill}%
\pgfsetlinewidth{0.501875pt}%
\definecolor{currentstroke}{rgb}{0.000000,0.000000,0.000000}%
\pgfsetstrokecolor{currentstroke}%
\pgfsetdash{}{0pt}%
\pgfsys@defobject{currentmarker}{\pgfqpoint{0.000000in}{-0.055556in}}{\pgfqpoint{0.000000in}{0.000000in}}{%
\pgfpathmoveto{\pgfqpoint{0.000000in}{0.000000in}}%
\pgfpathlineto{\pgfqpoint{0.000000in}{-0.055556in}}%
\pgfusepath{stroke,fill}%
}%
\begin{pgfscope}%
\pgfsys@transformshift{3.805790in}{2.148139in}%
\pgfsys@useobject{currentmarker}{}%
\end{pgfscope}%
\end{pgfscope}%
\begin{pgfscope}%
\pgftext[x=3.805790in,y=0.434277in,,top]{{\rmfamily\fontsize{11.000000}{13.200000}\selectfont 2.0}}%
\end{pgfscope}%
\begin{pgfscope}%
\pgfsetbuttcap%
\pgfsetroundjoin%
\definecolor{currentfill}{rgb}{0.000000,0.000000,0.000000}%
\pgfsetfillcolor{currentfill}%
\pgfsetlinewidth{0.501875pt}%
\definecolor{currentstroke}{rgb}{0.000000,0.000000,0.000000}%
\pgfsetstrokecolor{currentstroke}%
\pgfsetdash{}{0pt}%
\pgfsys@defobject{currentmarker}{\pgfqpoint{0.000000in}{0.000000in}}{\pgfqpoint{0.000000in}{0.055556in}}{%
\pgfpathmoveto{\pgfqpoint{0.000000in}{0.000000in}}%
\pgfpathlineto{\pgfqpoint{0.000000in}{0.055556in}}%
\pgfusepath{stroke,fill}%
}%
\begin{pgfscope}%
\pgfsys@transformshift{4.411336in}{0.489833in}%
\pgfsys@useobject{currentmarker}{}%
\end{pgfscope}%
\end{pgfscope}%
\begin{pgfscope}%
\pgfsetbuttcap%
\pgfsetroundjoin%
\definecolor{currentfill}{rgb}{0.000000,0.000000,0.000000}%
\pgfsetfillcolor{currentfill}%
\pgfsetlinewidth{0.501875pt}%
\definecolor{currentstroke}{rgb}{0.000000,0.000000,0.000000}%
\pgfsetstrokecolor{currentstroke}%
\pgfsetdash{}{0pt}%
\pgfsys@defobject{currentmarker}{\pgfqpoint{0.000000in}{-0.055556in}}{\pgfqpoint{0.000000in}{0.000000in}}{%
\pgfpathmoveto{\pgfqpoint{0.000000in}{0.000000in}}%
\pgfpathlineto{\pgfqpoint{0.000000in}{-0.055556in}}%
\pgfusepath{stroke,fill}%
}%
\begin{pgfscope}%
\pgfsys@transformshift{4.411336in}{2.148139in}%
\pgfsys@useobject{currentmarker}{}%
\end{pgfscope}%
\end{pgfscope}%
\begin{pgfscope}%
\pgftext[x=4.411336in,y=0.434277in,,top]{{\rmfamily\fontsize{11.000000}{13.200000}\selectfont 2.1}}%
\end{pgfscope}%
\begin{pgfscope}%
\pgfsetbuttcap%
\pgfsetroundjoin%
\definecolor{currentfill}{rgb}{0.000000,0.000000,0.000000}%
\pgfsetfillcolor{currentfill}%
\pgfsetlinewidth{0.501875pt}%
\definecolor{currentstroke}{rgb}{0.000000,0.000000,0.000000}%
\pgfsetstrokecolor{currentstroke}%
\pgfsetdash{}{0pt}%
\pgfsys@defobject{currentmarker}{\pgfqpoint{0.000000in}{0.000000in}}{\pgfqpoint{0.000000in}{0.055556in}}{%
\pgfpathmoveto{\pgfqpoint{0.000000in}{0.000000in}}%
\pgfpathlineto{\pgfqpoint{0.000000in}{0.055556in}}%
\pgfusepath{stroke,fill}%
}%
\begin{pgfscope}%
\pgfsys@transformshift{5.016881in}{0.489833in}%
\pgfsys@useobject{currentmarker}{}%
\end{pgfscope}%
\end{pgfscope}%
\begin{pgfscope}%
\pgfsetbuttcap%
\pgfsetroundjoin%
\definecolor{currentfill}{rgb}{0.000000,0.000000,0.000000}%
\pgfsetfillcolor{currentfill}%
\pgfsetlinewidth{0.501875pt}%
\definecolor{currentstroke}{rgb}{0.000000,0.000000,0.000000}%
\pgfsetstrokecolor{currentstroke}%
\pgfsetdash{}{0pt}%
\pgfsys@defobject{currentmarker}{\pgfqpoint{0.000000in}{-0.055556in}}{\pgfqpoint{0.000000in}{0.000000in}}{%
\pgfpathmoveto{\pgfqpoint{0.000000in}{0.000000in}}%
\pgfpathlineto{\pgfqpoint{0.000000in}{-0.055556in}}%
\pgfusepath{stroke,fill}%
}%
\begin{pgfscope}%
\pgfsys@transformshift{5.016881in}{2.148139in}%
\pgfsys@useobject{currentmarker}{}%
\end{pgfscope}%
\end{pgfscope}%
\begin{pgfscope}%
\pgftext[x=5.016881in,y=0.434277in,,top]{{\rmfamily\fontsize{11.000000}{13.200000}\selectfont 2.2}}%
\end{pgfscope}%
\begin{pgfscope}%
\pgfsetbuttcap%
\pgfsetroundjoin%
\definecolor{currentfill}{rgb}{0.000000,0.000000,0.000000}%
\pgfsetfillcolor{currentfill}%
\pgfsetlinewidth{0.501875pt}%
\definecolor{currentstroke}{rgb}{0.000000,0.000000,0.000000}%
\pgfsetstrokecolor{currentstroke}%
\pgfsetdash{}{0pt}%
\pgfsys@defobject{currentmarker}{\pgfqpoint{0.000000in}{0.000000in}}{\pgfqpoint{0.000000in}{0.055556in}}{%
\pgfpathmoveto{\pgfqpoint{0.000000in}{0.000000in}}%
\pgfpathlineto{\pgfqpoint{0.000000in}{0.055556in}}%
\pgfusepath{stroke,fill}%
}%
\begin{pgfscope}%
\pgfsys@transformshift{5.622427in}{0.489833in}%
\pgfsys@useobject{currentmarker}{}%
\end{pgfscope}%
\end{pgfscope}%
\begin{pgfscope}%
\pgfsetbuttcap%
\pgfsetroundjoin%
\definecolor{currentfill}{rgb}{0.000000,0.000000,0.000000}%
\pgfsetfillcolor{currentfill}%
\pgfsetlinewidth{0.501875pt}%
\definecolor{currentstroke}{rgb}{0.000000,0.000000,0.000000}%
\pgfsetstrokecolor{currentstroke}%
\pgfsetdash{}{0pt}%
\pgfsys@defobject{currentmarker}{\pgfqpoint{0.000000in}{-0.055556in}}{\pgfqpoint{0.000000in}{0.000000in}}{%
\pgfpathmoveto{\pgfqpoint{0.000000in}{0.000000in}}%
\pgfpathlineto{\pgfqpoint{0.000000in}{-0.055556in}}%
\pgfusepath{stroke,fill}%
}%
\begin{pgfscope}%
\pgfsys@transformshift{5.622427in}{2.148139in}%
\pgfsys@useobject{currentmarker}{}%
\end{pgfscope}%
\end{pgfscope}%
\begin{pgfscope}%
\pgftext[x=5.622427in,y=0.434277in,,top]{{\rmfamily\fontsize{11.000000}{13.200000}\selectfont 2.3}}%
\end{pgfscope}%
\begin{pgfscope}%
\pgfsetbuttcap%
\pgfsetroundjoin%
\definecolor{currentfill}{rgb}{0.000000,0.000000,0.000000}%
\pgfsetfillcolor{currentfill}%
\pgfsetlinewidth{0.501875pt}%
\definecolor{currentstroke}{rgb}{0.000000,0.000000,0.000000}%
\pgfsetstrokecolor{currentstroke}%
\pgfsetdash{}{0pt}%
\pgfsys@defobject{currentmarker}{\pgfqpoint{0.000000in}{0.000000in}}{\pgfqpoint{0.000000in}{0.055556in}}{%
\pgfpathmoveto{\pgfqpoint{0.000000in}{0.000000in}}%
\pgfpathlineto{\pgfqpoint{0.000000in}{0.055556in}}%
\pgfusepath{stroke,fill}%
}%
\begin{pgfscope}%
\pgfsys@transformshift{6.227972in}{0.489833in}%
\pgfsys@useobject{currentmarker}{}%
\end{pgfscope}%
\end{pgfscope}%
\begin{pgfscope}%
\pgfsetbuttcap%
\pgfsetroundjoin%
\definecolor{currentfill}{rgb}{0.000000,0.000000,0.000000}%
\pgfsetfillcolor{currentfill}%
\pgfsetlinewidth{0.501875pt}%
\definecolor{currentstroke}{rgb}{0.000000,0.000000,0.000000}%
\pgfsetstrokecolor{currentstroke}%
\pgfsetdash{}{0pt}%
\pgfsys@defobject{currentmarker}{\pgfqpoint{0.000000in}{-0.055556in}}{\pgfqpoint{0.000000in}{0.000000in}}{%
\pgfpathmoveto{\pgfqpoint{0.000000in}{0.000000in}}%
\pgfpathlineto{\pgfqpoint{0.000000in}{-0.055556in}}%
\pgfusepath{stroke,fill}%
}%
\begin{pgfscope}%
\pgfsys@transformshift{6.227972in}{2.148139in}%
\pgfsys@useobject{currentmarker}{}%
\end{pgfscope}%
\end{pgfscope}%
\begin{pgfscope}%
\pgftext[x=6.227972in,y=0.434277in,,top]{{\rmfamily\fontsize{11.000000}{13.200000}\selectfont 2.4}}%
\end{pgfscope}%
\begin{pgfscope}%
\pgftext[x=5.016881in,y=0.229166in,,top]{{\rmfamily\fontsize{11.000000}{13.200000}\selectfont \(\displaystyle k_BT\)\quad[ \(\displaystyle J\) ]}}%
\end{pgfscope}%
\begin{pgfscope}%
\pgfsetbuttcap%
\pgfsetroundjoin%
\definecolor{currentfill}{rgb}{0.000000,0.000000,0.000000}%
\pgfsetfillcolor{currentfill}%
\pgfsetlinewidth{0.501875pt}%
\definecolor{currentstroke}{rgb}{0.000000,0.000000,0.000000}%
\pgfsetstrokecolor{currentstroke}%
\pgfsetdash{}{0pt}%
\pgfsys@defobject{currentmarker}{\pgfqpoint{0.000000in}{0.000000in}}{\pgfqpoint{0.055556in}{0.000000in}}{%
\pgfpathmoveto{\pgfqpoint{0.000000in}{0.000000in}}%
\pgfpathlineto{\pgfqpoint{0.055556in}{0.000000in}}%
\pgfusepath{stroke,fill}%
}%
\begin{pgfscope}%
\pgfsys@transformshift{3.805790in}{0.489833in}%
\pgfsys@useobject{currentmarker}{}%
\end{pgfscope}%
\end{pgfscope}%
\begin{pgfscope}%
\pgfsetbuttcap%
\pgfsetroundjoin%
\definecolor{currentfill}{rgb}{0.000000,0.000000,0.000000}%
\pgfsetfillcolor{currentfill}%
\pgfsetlinewidth{0.501875pt}%
\definecolor{currentstroke}{rgb}{0.000000,0.000000,0.000000}%
\pgfsetstrokecolor{currentstroke}%
\pgfsetdash{}{0pt}%
\pgfsys@defobject{currentmarker}{\pgfqpoint{-0.055556in}{0.000000in}}{\pgfqpoint{0.000000in}{0.000000in}}{%
\pgfpathmoveto{\pgfqpoint{0.000000in}{0.000000in}}%
\pgfpathlineto{\pgfqpoint{-0.055556in}{0.000000in}}%
\pgfusepath{stroke,fill}%
}%
\begin{pgfscope}%
\pgfsys@transformshift{6.227972in}{0.489833in}%
\pgfsys@useobject{currentmarker}{}%
\end{pgfscope}%
\end{pgfscope}%
\begin{pgfscope}%
\pgftext[x=3.750235in,y=0.489833in,right,]{{\rmfamily\fontsize{11.000000}{13.200000}\selectfont 0}}%
\end{pgfscope}%
\begin{pgfscope}%
\pgfsetbuttcap%
\pgfsetroundjoin%
\definecolor{currentfill}{rgb}{0.000000,0.000000,0.000000}%
\pgfsetfillcolor{currentfill}%
\pgfsetlinewidth{0.501875pt}%
\definecolor{currentstroke}{rgb}{0.000000,0.000000,0.000000}%
\pgfsetstrokecolor{currentstroke}%
\pgfsetdash{}{0pt}%
\pgfsys@defobject{currentmarker}{\pgfqpoint{0.000000in}{0.000000in}}{\pgfqpoint{0.055556in}{0.000000in}}{%
\pgfpathmoveto{\pgfqpoint{0.000000in}{0.000000in}}%
\pgfpathlineto{\pgfqpoint{0.055556in}{0.000000in}}%
\pgfusepath{stroke,fill}%
}%
\begin{pgfscope}%
\pgfsys@transformshift{3.805790in}{0.766217in}%
\pgfsys@useobject{currentmarker}{}%
\end{pgfscope}%
\end{pgfscope}%
\begin{pgfscope}%
\pgfsetbuttcap%
\pgfsetroundjoin%
\definecolor{currentfill}{rgb}{0.000000,0.000000,0.000000}%
\pgfsetfillcolor{currentfill}%
\pgfsetlinewidth{0.501875pt}%
\definecolor{currentstroke}{rgb}{0.000000,0.000000,0.000000}%
\pgfsetstrokecolor{currentstroke}%
\pgfsetdash{}{0pt}%
\pgfsys@defobject{currentmarker}{\pgfqpoint{-0.055556in}{0.000000in}}{\pgfqpoint{0.000000in}{0.000000in}}{%
\pgfpathmoveto{\pgfqpoint{0.000000in}{0.000000in}}%
\pgfpathlineto{\pgfqpoint{-0.055556in}{0.000000in}}%
\pgfusepath{stroke,fill}%
}%
\begin{pgfscope}%
\pgfsys@transformshift{6.227972in}{0.766217in}%
\pgfsys@useobject{currentmarker}{}%
\end{pgfscope}%
\end{pgfscope}%
\begin{pgfscope}%
\pgftext[x=3.750235in,y=0.766217in,right,]{{\rmfamily\fontsize{11.000000}{13.200000}\selectfont 50}}%
\end{pgfscope}%
\begin{pgfscope}%
\pgfsetbuttcap%
\pgfsetroundjoin%
\definecolor{currentfill}{rgb}{0.000000,0.000000,0.000000}%
\pgfsetfillcolor{currentfill}%
\pgfsetlinewidth{0.501875pt}%
\definecolor{currentstroke}{rgb}{0.000000,0.000000,0.000000}%
\pgfsetstrokecolor{currentstroke}%
\pgfsetdash{}{0pt}%
\pgfsys@defobject{currentmarker}{\pgfqpoint{0.000000in}{0.000000in}}{\pgfqpoint{0.055556in}{0.000000in}}{%
\pgfpathmoveto{\pgfqpoint{0.000000in}{0.000000in}}%
\pgfpathlineto{\pgfqpoint{0.055556in}{0.000000in}}%
\pgfusepath{stroke,fill}%
}%
\begin{pgfscope}%
\pgfsys@transformshift{3.805790in}{1.042602in}%
\pgfsys@useobject{currentmarker}{}%
\end{pgfscope}%
\end{pgfscope}%
\begin{pgfscope}%
\pgfsetbuttcap%
\pgfsetroundjoin%
\definecolor{currentfill}{rgb}{0.000000,0.000000,0.000000}%
\pgfsetfillcolor{currentfill}%
\pgfsetlinewidth{0.501875pt}%
\definecolor{currentstroke}{rgb}{0.000000,0.000000,0.000000}%
\pgfsetstrokecolor{currentstroke}%
\pgfsetdash{}{0pt}%
\pgfsys@defobject{currentmarker}{\pgfqpoint{-0.055556in}{0.000000in}}{\pgfqpoint{0.000000in}{0.000000in}}{%
\pgfpathmoveto{\pgfqpoint{0.000000in}{0.000000in}}%
\pgfpathlineto{\pgfqpoint{-0.055556in}{0.000000in}}%
\pgfusepath{stroke,fill}%
}%
\begin{pgfscope}%
\pgfsys@transformshift{6.227972in}{1.042602in}%
\pgfsys@useobject{currentmarker}{}%
\end{pgfscope}%
\end{pgfscope}%
\begin{pgfscope}%
\pgftext[x=3.750235in,y=1.042602in,right,]{{\rmfamily\fontsize{11.000000}{13.200000}\selectfont 100}}%
\end{pgfscope}%
\begin{pgfscope}%
\pgfsetbuttcap%
\pgfsetroundjoin%
\definecolor{currentfill}{rgb}{0.000000,0.000000,0.000000}%
\pgfsetfillcolor{currentfill}%
\pgfsetlinewidth{0.501875pt}%
\definecolor{currentstroke}{rgb}{0.000000,0.000000,0.000000}%
\pgfsetstrokecolor{currentstroke}%
\pgfsetdash{}{0pt}%
\pgfsys@defobject{currentmarker}{\pgfqpoint{0.000000in}{0.000000in}}{\pgfqpoint{0.055556in}{0.000000in}}{%
\pgfpathmoveto{\pgfqpoint{0.000000in}{0.000000in}}%
\pgfpathlineto{\pgfqpoint{0.055556in}{0.000000in}}%
\pgfusepath{stroke,fill}%
}%
\begin{pgfscope}%
\pgfsys@transformshift{3.805790in}{1.318986in}%
\pgfsys@useobject{currentmarker}{}%
\end{pgfscope}%
\end{pgfscope}%
\begin{pgfscope}%
\pgfsetbuttcap%
\pgfsetroundjoin%
\definecolor{currentfill}{rgb}{0.000000,0.000000,0.000000}%
\pgfsetfillcolor{currentfill}%
\pgfsetlinewidth{0.501875pt}%
\definecolor{currentstroke}{rgb}{0.000000,0.000000,0.000000}%
\pgfsetstrokecolor{currentstroke}%
\pgfsetdash{}{0pt}%
\pgfsys@defobject{currentmarker}{\pgfqpoint{-0.055556in}{0.000000in}}{\pgfqpoint{0.000000in}{0.000000in}}{%
\pgfpathmoveto{\pgfqpoint{0.000000in}{0.000000in}}%
\pgfpathlineto{\pgfqpoint{-0.055556in}{0.000000in}}%
\pgfusepath{stroke,fill}%
}%
\begin{pgfscope}%
\pgfsys@transformshift{6.227972in}{1.318986in}%
\pgfsys@useobject{currentmarker}{}%
\end{pgfscope}%
\end{pgfscope}%
\begin{pgfscope}%
\pgftext[x=3.750235in,y=1.318986in,right,]{{\rmfamily\fontsize{11.000000}{13.200000}\selectfont 150}}%
\end{pgfscope}%
\begin{pgfscope}%
\pgfsetbuttcap%
\pgfsetroundjoin%
\definecolor{currentfill}{rgb}{0.000000,0.000000,0.000000}%
\pgfsetfillcolor{currentfill}%
\pgfsetlinewidth{0.501875pt}%
\definecolor{currentstroke}{rgb}{0.000000,0.000000,0.000000}%
\pgfsetstrokecolor{currentstroke}%
\pgfsetdash{}{0pt}%
\pgfsys@defobject{currentmarker}{\pgfqpoint{0.000000in}{0.000000in}}{\pgfqpoint{0.055556in}{0.000000in}}{%
\pgfpathmoveto{\pgfqpoint{0.000000in}{0.000000in}}%
\pgfpathlineto{\pgfqpoint{0.055556in}{0.000000in}}%
\pgfusepath{stroke,fill}%
}%
\begin{pgfscope}%
\pgfsys@transformshift{3.805790in}{1.595370in}%
\pgfsys@useobject{currentmarker}{}%
\end{pgfscope}%
\end{pgfscope}%
\begin{pgfscope}%
\pgfsetbuttcap%
\pgfsetroundjoin%
\definecolor{currentfill}{rgb}{0.000000,0.000000,0.000000}%
\pgfsetfillcolor{currentfill}%
\pgfsetlinewidth{0.501875pt}%
\definecolor{currentstroke}{rgb}{0.000000,0.000000,0.000000}%
\pgfsetstrokecolor{currentstroke}%
\pgfsetdash{}{0pt}%
\pgfsys@defobject{currentmarker}{\pgfqpoint{-0.055556in}{0.000000in}}{\pgfqpoint{0.000000in}{0.000000in}}{%
\pgfpathmoveto{\pgfqpoint{0.000000in}{0.000000in}}%
\pgfpathlineto{\pgfqpoint{-0.055556in}{0.000000in}}%
\pgfusepath{stroke,fill}%
}%
\begin{pgfscope}%
\pgfsys@transformshift{6.227972in}{1.595370in}%
\pgfsys@useobject{currentmarker}{}%
\end{pgfscope}%
\end{pgfscope}%
\begin{pgfscope}%
\pgftext[x=3.750235in,y=1.595370in,right,]{{\rmfamily\fontsize{11.000000}{13.200000}\selectfont 200}}%
\end{pgfscope}%
\begin{pgfscope}%
\pgfsetbuttcap%
\pgfsetroundjoin%
\definecolor{currentfill}{rgb}{0.000000,0.000000,0.000000}%
\pgfsetfillcolor{currentfill}%
\pgfsetlinewidth{0.501875pt}%
\definecolor{currentstroke}{rgb}{0.000000,0.000000,0.000000}%
\pgfsetstrokecolor{currentstroke}%
\pgfsetdash{}{0pt}%
\pgfsys@defobject{currentmarker}{\pgfqpoint{0.000000in}{0.000000in}}{\pgfqpoint{0.055556in}{0.000000in}}{%
\pgfpathmoveto{\pgfqpoint{0.000000in}{0.000000in}}%
\pgfpathlineto{\pgfqpoint{0.055556in}{0.000000in}}%
\pgfusepath{stroke,fill}%
}%
\begin{pgfscope}%
\pgfsys@transformshift{3.805790in}{1.871755in}%
\pgfsys@useobject{currentmarker}{}%
\end{pgfscope}%
\end{pgfscope}%
\begin{pgfscope}%
\pgfsetbuttcap%
\pgfsetroundjoin%
\definecolor{currentfill}{rgb}{0.000000,0.000000,0.000000}%
\pgfsetfillcolor{currentfill}%
\pgfsetlinewidth{0.501875pt}%
\definecolor{currentstroke}{rgb}{0.000000,0.000000,0.000000}%
\pgfsetstrokecolor{currentstroke}%
\pgfsetdash{}{0pt}%
\pgfsys@defobject{currentmarker}{\pgfqpoint{-0.055556in}{0.000000in}}{\pgfqpoint{0.000000in}{0.000000in}}{%
\pgfpathmoveto{\pgfqpoint{0.000000in}{0.000000in}}%
\pgfpathlineto{\pgfqpoint{-0.055556in}{0.000000in}}%
\pgfusepath{stroke,fill}%
}%
\begin{pgfscope}%
\pgfsys@transformshift{6.227972in}{1.871755in}%
\pgfsys@useobject{currentmarker}{}%
\end{pgfscope}%
\end{pgfscope}%
\begin{pgfscope}%
\pgftext[x=3.750235in,y=1.871755in,right,]{{\rmfamily\fontsize{11.000000}{13.200000}\selectfont 250}}%
\end{pgfscope}%
\begin{pgfscope}%
\pgfsetbuttcap%
\pgfsetroundjoin%
\definecolor{currentfill}{rgb}{0.000000,0.000000,0.000000}%
\pgfsetfillcolor{currentfill}%
\pgfsetlinewidth{0.501875pt}%
\definecolor{currentstroke}{rgb}{0.000000,0.000000,0.000000}%
\pgfsetstrokecolor{currentstroke}%
\pgfsetdash{}{0pt}%
\pgfsys@defobject{currentmarker}{\pgfqpoint{0.000000in}{0.000000in}}{\pgfqpoint{0.055556in}{0.000000in}}{%
\pgfpathmoveto{\pgfqpoint{0.000000in}{0.000000in}}%
\pgfpathlineto{\pgfqpoint{0.055556in}{0.000000in}}%
\pgfusepath{stroke,fill}%
}%
\begin{pgfscope}%
\pgfsys@transformshift{3.805790in}{2.148139in}%
\pgfsys@useobject{currentmarker}{}%
\end{pgfscope}%
\end{pgfscope}%
\begin{pgfscope}%
\pgfsetbuttcap%
\pgfsetroundjoin%
\definecolor{currentfill}{rgb}{0.000000,0.000000,0.000000}%
\pgfsetfillcolor{currentfill}%
\pgfsetlinewidth{0.501875pt}%
\definecolor{currentstroke}{rgb}{0.000000,0.000000,0.000000}%
\pgfsetstrokecolor{currentstroke}%
\pgfsetdash{}{0pt}%
\pgfsys@defobject{currentmarker}{\pgfqpoint{-0.055556in}{0.000000in}}{\pgfqpoint{0.000000in}{0.000000in}}{%
\pgfpathmoveto{\pgfqpoint{0.000000in}{0.000000in}}%
\pgfpathlineto{\pgfqpoint{-0.055556in}{0.000000in}}%
\pgfusepath{stroke,fill}%
}%
\begin{pgfscope}%
\pgfsys@transformshift{6.227972in}{2.148139in}%
\pgfsys@useobject{currentmarker}{}%
\end{pgfscope}%
\end{pgfscope}%
\begin{pgfscope}%
\pgftext[x=3.750235in,y=2.148139in,right,]{{\rmfamily\fontsize{11.000000}{13.200000}\selectfont 300}}%
\end{pgfscope}%
\begin{pgfscope}%
\pgftext[x=3.456207in,y=1.318986in,,bottom,rotate=90.000000]{{\rmfamily\fontsize{11.000000}{13.200000}\selectfont \(\displaystyle C_V/k_B\)\quad[ \(\displaystyle \cdot\) ]}}%
\end{pgfscope}%
\begin{pgfscope}%
\pgftext[x=5.016881in,y=2.217583in,,base]{{\rmfamily\fontsize{11.000000}{13.200000}\selectfont Heat capacity vs thermal energy}}%
\end{pgfscope}%
\end{pgfpicture}%
\makeatother%
\endgroup%

> I Present your results\\
> I Give a critical discussion of your work and place it in the correct context.\\
> I Relate your work to other calculations/studies\\
> I An eventual reader should be able to reproduce your calculations if she/he wants to do so. All input variables should be properly explained.\\
> I Make sure that figures and tables should contain enough information in their captions, axis labels etc so that an eventual reader can gain a first impression of your work by studying figures and tables only.
\section*{\uppercase{Conclusion and perspectives}}
> I State your main findings and interpretations\\
> I Try as far as possible to present perspectives for future work\\
> I Try to discuss the pros and cons of the methods and possible improvements

\printbibliography
\end{document}